\section{RG Flow and Computational Behavior}
\label{sec:rg-flow}

§3 equips §4 with $\beta$/$\gamma$ functions so renormalized correlators are scale-invariant.

Having established the regulator framework in Section~\ref{sec:regularization} and the scattering semantics in Section~\ref{sec:scattering-invitation}, we now develop the renormalisation group machinery \cite{kadanoff1966,wilson1971,wilson1974} that reveals how computational processes evolve toward fixed points. In the physics domain, this RG flow controls how scattering amplitudes evolve with energy scale.

This section builds on the regulator hierarchy, normalisation operator, generating function $\mathcal{G}(z,\bar{z};\vec{q},\Lambda)$, and Green's functions from previous sections.

\paragraph{RG flow observables vs renormalisation conditions}
We distinguish between RG flow observables (evolve under RG): $\vec{q}_t$, $\mathcal{G}_t$, $\Lambda_t$, $\mathcal{Z}_{n,m}(\vec{q}_t)$ and renormalisation conditions (fixed): $\tau$ (termination threshold), normalisation scheme choices.

The relationship between scale behaviour and computational reversibility can be understood through how the scale parameter $\Lambda$ interacts with the three grading parameters $\vec{q} = (q_1,q_2,q_3)$ in our generating function from equation~\eqref{eq:generating-function-g6}. See the canonical domain ledger Table~\ref{tab:universal-domain-translation} for the single cross-domain dictionary referenced throughout.

\subsection{Dimensionless Couplings and RG Flow}

We work with dimensionless couplings $g_i(\mu)$ where $\mu$ is a fixed reference scale. This avoids the $\beta_\Lambda$ contradiction that would arise from including the scale parameter in fixed-point conditions.

\begin{definition}[Dimensionless Couplings]
\label{def:dimensionless-couplings}
Let $\mu$ be a fixed reference scale. Define dimensionless couplings:
\[
g_i(\mu) := \frac{\mathcal{Z}_{i,0}(\vec{q})}{\mu^{d_i}}
\]
where $d_i$ are the canonical dimensions of the correlators. The RG flow equations become:
\[
\frac{dg_i}{dt} = \beta_i(\{g_j\}), \quad t = \log(\Lambda/\mu)
\]
where $\beta_i$ depend only on the dimensionless couplings $\{g_j\}$.
\end{definition}

\begin{definition}[RG Flow Behavior Classification]
\label{def:rg-flow-classification}
Fix admissible $\mathcal{A}$ and qmask; compare in that quotient. Computational systems exhibit different behaviours under RG flow:

\begin{table}[h]
\centering
\begin{tabular}{|l|l|}
\hline
\textbf{RG Behavior} & \textbf{Computational Reading} \\
\hline
Converging flow & Reversible computation \\
Diverging flow & Irreversible computation \\
Marginal flow & Undecidable computation \\
\hline
\end{tabular}
\caption{RG flow behaviours and computational interpretations}
\end{table}
\end{definition}

\begin{notation}[Assumptions]
\label{not:assumptions}
\textbf{Assumptions:} Positivity $\mathcal{Z}_{n,m}\ge0$; contractivity of $\mathcal{R}_\Lambda$; qmask comparisons. These assumptions are stated once and referenced throughout this section.
\end{notation}

\subsection{RG Operator and $\beta$-System}

\begin{definition}[RG Operator and $\beta$-System]
\label{def:rg-operator-beta-system}
The general RG flow operator is a family of maps $\mathcal{R}_t: \mathcal{F} \times \mathcal{M} \to \mathcal{F} \times \mathcal{M}$ parameterized by RG time $t \in \mathbb{R}$, with semigroup property $\mathcal{R}_{t+s} = \mathcal{R}_t \circ \mathcal{R}_s$. The RG flow equations are:
\begin{align}
\frac{d\mathcal{G}_t}{dt} &= \beta_{\mathcal{G}}(\mathcal{G}_t, \vec{a}_t) \\
\frac{d\vec{a}_t}{dt} &= \vec{\beta}_a(\mathcal{G}_t, \vec{a}_t) \\
\frac{dg_k}{dt} &= \beta_k(\{g_j\}), \quad t = \log(\Lambda/\mu)
\end{align}
where $\beta_{\mathcal{G}}$, $\vec{\beta}_a$, and $\beta_k$ are the beta functions for generating function, moduli, and dimensionless couplings respectively.
\end{definition}

\begin{proposition}[Entropy Monotonicity (Model-dependent)]
\label{thm:entropy-monotonicity}
\textbf{Assumptions:} $\mathcal{Z}_{n,m}\ge0$; $\mathcal{R}_\Lambda$ is contractive/Markov on $p_{n,m}$; $t=\log\Lambda$.
Under RG flow, the entropy is non-increasing: $\frac{dS}{dt} \leq 0$. Equality iff reversible.
\textbf{Proof sketch:} Entropy monotonicity → direction.

\textbf{Physics interpretation:} In scattering theory, this corresponds to the optical theorem ensuring forward-channel dominance as energy increases.
\end{proposition}

\subsection{Single Observable $\mathcal{O}(\Lambda)$}

Fix $\vec{q}$. Define $\mathcal{O}(\Lambda):=\sum_{n,m\ge0}\mathcal{Z}_{n,m}(\vec{q})\,\Lambda^{-(n+m)}$ as the only running scalar. Its beta-function is $\beta_\mathcal{O}(\Lambda):=\frac{d}{d\log\Lambda}\mathcal{O}(\Lambda)$.

\begin{proposition}[Indicative RG–computational correspondence]
\label{prop:rg-corresp}
Assume $\mathcal{Z}_{n,m}\ge0$ and $\mathcal{R}_\Lambda$ decreases $(n+m)$-weight.
Then $\beta_\mathcal{O}\le0$ and convergence of $\mathcal{O}(\Lambda)$ implies
no information loss under the flow (reversible fragment). \textbf{Directionality:} logic inconsistency $\Rightarrow$ domain divergence (not conversely).
\end{proposition}

\begin{conjecture}[Marginal flow and undecidability]
\label{conj:marginal}
Marginal behaviour of $\mathcal{O}$ corresponds to boundary cases where halting cannot
be decided within the base fragment $\mathcal{L}_0$.
\end{conjecture}

\begin{example}[Synthetic RG Flow]
\label{ex:synthetic-rg-flow}
Consider a parameter $g$ in our type graph that drifts under coarse-graining. As $\Lambda$ increases, the observable $\mathcal{O}(\Lambda)$ evolves according to:
\[
\mathcal{O}(\Lambda) = g_0 \Lambda^{-\gamma} + \text{subleading terms}
\]
where $\gamma$ is the anomalous dimension. This gives $\beta(g) = -\gamma g$, showing how the parameter flows under RG evolution.
\end{example}

\subsection{Callan–Symanzik Equation as Trace}

The CS trace of the flow (Def.~\ref{def:rg-operator-beta-system}) vanishes. The exact trace form is:
\[
\Big(\Lambda\partial_\Lambda + \sum_i \beta_i\partial_{q_i} - \gamma\Big) G(z,\bar z;\vec q,\Lambda)=0.
\]
Reference Appendix C for derivation; $\gamma$ is the anomalous dimension with sign convention as in standard RG theory.

\begin{notation}[Optical identity as CS‑trace]
\label{not:optical-identity}
The optical identity connects forward-channel unitarity to the CS equation:
\[
\sum_\alpha|P_\alpha S_N\psi|^2=|\psi|^2 \Longrightarrow \big(\Lambda\partial_\Lambda+\beta_i\partial_{q_i}-\gamma\big)G=0\ \text{on forward channel.}
\]
\end{notation}

\subsection{RG Fixed Points and Universality Classes}

\begin{definition}[RG Fixed Points]
\label{def:rg-fixed-points}
An RG fixed point is where all beta functions for dimensionless couplings vanish: $\beta_i = 0$ for all $i$. This defines scale-invariant points where the dimensionless couplings $g_i(\mu)$ are constant under RG flow.
\end{definition}

\begin{definition}[Computational Universality Classes]
\label{def:computational-universality}
Systems belong to universality classes based on RG flow behaviour:
\begin{itemize}
\item Class I: Converging RG flow (reversible computation)
\item Class II: Diverging RG flow (irreversible computation)  
\item Class III: Marginal RG flow (undecidable computation)
\end{itemize}
\end{definition}

\subsection{AGT Connection (Interpretation)}
\begin{remark}
\textbf{Interpretation:} For conditional AGT identifications and representation-theory prerequisites, see Section 11. Not used in proofs.
\end{remark}

\subsection{Summary and Outlook}

This section established RG flow as a truth predicate for computation:

\begin{enumerate}
\item RG flow behaviour classifies systems into three universality classes
\item Converging flow $\to$ reversible computation (truth) $\leftrightarrow$ converging flow and forward‑channel optical identity
\item Diverging flow $\to$ irreversible computation (falsehood)  
\item Marginal flow $\to$ undecidable computation
\end{enumerate}

The RG framework reveals deep connections between computation, information theory, and physics. The next section develops the complete renormalisation procedure (Section~\ref{sec:renormalisation}).

Adapter back to §1: the flow $(\Lambda,\vec{q},z,\bar{z})\mapsto (b\Lambda,\vec{q}(b),b^{-1}z,b^{-1}\bar{z})$ preserves the parameter map fixed in §1.

\paragraph{Ledger note.} Every beta function, c-function, or universality class that appears in this section has a row on the computation ledger: the same algebraic data will resurface as consistency witnesses in Sections~\ref{sec:truth-fixed-point}--\ref{sec:consistency} and again when we specialise to the LLM and spectral domains. This keeps the logical core independent of the physical metaphors used here.

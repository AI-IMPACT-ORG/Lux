\section{The Regulator View: Understanding Compilation}
\label{sec:regularization}

Having established the basic computational framework in Section~\ref{sec:computation-paradigms}, we now explore how regularization provides the mathematical foundation for making computational processes well-defined.

This section builds on the $\mathsf{Gen4}$ primitive, paradigm parameters $\vec{q}$, scale parameter $\Lambda$, and the universal computational cycle from Section~\ref{sec:computation-paradigms}.

\begin{notation}[Formal vs Analytic Usage]
\label{not:formal-analytic}
\textbf{Critical Distinction:} Throughout Sections~\ref{sec:computation-paradigms}--\ref{sec:formal-systems}, $\mathcal{G}(z,\bar{z};\vec{q},\Lambda)$ is treated as a \textbf{formal power series}. Analytic claims (convergence, zeta-regularization, spectral properties) appear only in Sections~\ref{sec:unified-theory}--\ref{sec:spectral-gap} under additional hypotheses. This avoids category mistakes between formal algebraic objects and analytic functions.
\end{notation}

\paragraph{What this section adds}
We introduce regulator hierarchy, normalisation operator, termination condition, and moduli space of normalized programmes.

The normalisation step in our computational cycle corresponds precisely to choosing regulators that control the behaviour of the $\mathsf{Gen4}$ primitive within the L/B/R structure.

\subsection{Four-Moduli Parametric Normaliser}

We use the **four‑moduli parametric normaliser** ($NF_{\mu_L,\theta_L,\mu_R,\theta_R}$) acting **on headers only**, with left/right scale/phase flows; we refer to the CLASS spec for the exact definition and flow‑compatibility requirements. Here we only use that $NF$ is header‑only and boundary‑aware.

\subsection{The Regulator View of Computation in L/B/R Structure}

The computational process encoding $\to$ operator application $\to$ normalisation (regularization) $\to$ decoding corresponds directly to compilation within the L/B/R structure. The regularization map is:
\[
\mathcal{R}_\Lambda: \mathsf{Gen4} \mapsto \mathsf{Gen4}^{\text{reg}} = \sum_{n,m\ge0}\frac{z^n\bar{z}^{\,m}}{n!\,m!}\,\mathcal{Z}_{n,m}(\vec{q})\,\Lambda^{-(n+m)} \cdot e^{-(n+m)/K}
\]
where $K \in \mathbb{N}$ is the degree cutoff and $\Lambda \in \mathbb{R}_+$ is the physical scale. The smooth kernel $e^{-(n+m)/K}$ avoids scheme dependence issues.\footnote{Sharp $\Theta$-cutoff is equivalent at observational level.}

The parametric normalizer $NF_{\mu_L,\theta_L,\mu_R,\theta_R}$ couples moduli to normalisation, where $\mu_L, \theta_L, \mu_R, \theta_R \in L$ control scale and phase flows across boundaries.

\subsection{Regulator Hierarchy}

Our framework introduces a natural hierarchy of regulators that control the computational process, providing a systematic way to understand how different levels of regularization interact:

\begin{definition}[Regulator Hierarchy]
\label{def:regulator-hierarchy}
\textbf{Conceptual}: Boundary at $\infty$ (universal computation) \\
\textbf{Operational}: Scale parameter $\Lambda$ \\
\textbf{Grading}: Three grading parameters $\vec{q}$ \\
\textbf{State}: Virasoro levels $n, m$
\end{definition}

\paragraph{Syntactic vs semantic regulators:} Syntactic regulators control typing/arity and grammar constraints; semantic regulators control observable choice and evaluation budget. Write $\llbracket\phi\rrbracket_{\Lambda}^{\text{syn},\text{sem}}$ for evaluation with both regulator types.

\subsection{Termination Condition as Regularization}

A run halts at $T_{\text{halt}} := \inf\{t \geq 0 : \langle \psi(t)|P_{\mathrm{vac}}|\psi(t)\rangle \geq \tau\}$ for fixed $\tau\in(0,1]$. The stopping rule at threshold $\tau$ provides paradigm-independent normalisation criterion.

\subsection{Regularization Principles}

Regularization manifests differently across computational paradigms, but follows universal principles that transcend the specific implementation:

\begin{itemize}
\item Boundary conditions provide natural stopping criteria for computational processes
\item normalisation procedures ensure well-defined results across all paradigms
\item Scale parameters control the computational process through the generating function
\end{itemize}

\begin{definition}[Program → Coefficients Map and Moduli Stack]
\label{def:programme-coefficient-map}
The map from programmes to coefficients: $\mathcal{Z}_{n,m}(\vec{q}) = \sum_{\text{paths }p:\,|p|=(n,m)} w(p;\vec{q})$. The moduli space $\mathcal{M} := [\mathfrak{P} / \sim]$ where $(\text{programme},\text{data})\sim(\text{programme}',\text{data}')$ iff related by normalisation $N$ and RG-preserving isomorphisms.
\end{definition}

\subsection{Example: Peano universality across views}
\label{subsec:peano-moduli}
Let $\mathbf{PA}$ be the Lawvere theory of a natural numbers object (NNO)
with constants $0$ and successor $S$, and induction.

\paragraph{Four presentations.}
\begin{itemize}
\item $\mathbf{TM}$: a register machine for $S$ and primitive recursion on $\mathbb{N}$;
\item $\mathbf{\Lambda}$: Church numerals with $0,\;S,\;$ and a recursor $R$;
\item $\mathbf{Path}$: a path-integral model with amplitudes supported on unary steps $n\to n+1$;
\item $\mathbf{Circ}$: a symmetric monoidal presentation with generators $0,S$ and recursion combinators.
\end{itemize}

\paragraph{Encoders and normalisation.}
There are encoders $E_{i\to j}$ sending each presentation to another,
with $E_{j\to i}\circ E_{i\to j}\simeq N$.
All four presentations define the same point $[\mathbf{PA}]\in\mathcal{M}$.

\begin{conjecture}[Presentation Alignment]
\label{conj:alignment}
The four presentations $\mathbf{TM}$, $\mathbf{\Lambda}$, $\mathbf{Path}$, and $\mathbf{Circ}$ of Peano arithmetic are equivalent under encoder composition and normalisation. The encoders $E_{i\to j}$ preserve the computational semantics while transforming the syntactic presentation.
\end{conjecture}

\begin{proposition}[Peano equivalence up to normalisation]
Each presentation of $\mathbf{PA}$ yields the same normalized correlators for
the generating function $G$, i.e.\ $\mathcal{O}(\Lambda)$ (the global observable defined in Section~\ref{sec:rg-flow})
is invariant under encoders and normalisation. Hence all four belong to the same moduli point.
\end{proposition}

\subsection{Summary and Outlook}

This section established the regulator view of computation:

\begin{enumerate}
\item normalisation step corresponds to choosing appropriate regulators
\item Regulator hierarchy controls computational process at multiple scales
\item Termination condition provides paradigm-independent normalisation criterion
\item Traditional models are different regularization choices within our unified framework
\end{enumerate}

The regulator view unifies compile-time and runtime aspects through the generating function structure. The next section develops RG machinery showing how regulators evolve toward fixed points (Section~\ref{sec:rg-flow}).

\paragraph{Back to the computation ledger.} Whenever a regulator or observable is introduced here, we retain its computation-domain meaning: $\mathcal{R}_\Lambda$ is simultaneously a physical coarse-graining map and the logic transformer that will be used again when we discuss training dynamics (Section~\ref{sec:llm_rg}). We explicitly note these dual roles so that the logic statements that depend only on the computation ledger remain easy to identify.

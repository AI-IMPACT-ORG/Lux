\section{Truth as Fixed Point: RG Flow as Logical Semantics}
\label{sec:truth-fixed-point}

§6 uses §§3-5's RG framework and logic to define truth as RG fixed points.

Having established the interacting positive logic framework in Section~\ref{sec:formal-systems}, we now present the core innovation of this work: truth can be understood as a fixed point under renormalisation group flow. This section builds directly on the $\mathsf{Gen4}$ primitive from Section~\ref{sec:computation-paradigms}. In the physics domain, this corresponds to unitarity constraints that emerge as fixed points of the RG flow for S-matrix elements.

The connection between renormalisation and logic provides the foundation for our computational semantics. Everything in this section is domain-neutral; truth semantics implement the Green's functions hierarchy through convergence properties of the RG flow.

\subsection{Observable Metric and Contractivity}

\textbf{Proof sketch:} Kleene fixed point on $\omega$-cpo + monotone, $\omega$-continuous $\mathcal{R}$ $\Rightarrow$ lfp exists. \textbf{Details:} Metric and contractivity definitions moved to Appendix C.

\begin{definition}[Callan–Symanzik Equation]
\label{def:callan-symanzik}
The relationship between bare and renormalized correlators is expressed through the Callan–Symanzik equation \cite{callan1970,symanzik1970}:
\[
\left(\Lambda \frac{\partial}{\partial \Lambda} + \beta_{\mathsf{Gen4}} \frac{\partial}{\partial \mathsf{Gen4}} + \vec{\beta}_q \cdot \frac{\partial}{\partial \vec{q}} + \gamma\right) \mathsf{Gen4}^{\text{ren}} = 0
\]
where $\gamma$ is the anomalous dimension and $\vec{\beta}_q$ are the beta functions for the coupling parameters.
\end{definition}

The interacting positive logic provides a natural framework for understanding truth as fixed points. The L/B/R structure creates boundaries where truth can be evaluated, with the bulk providing the computational dynamics. For cross-domain interpretations, see the canonical ledger Table~\ref{tab:universal-domain-translation}.

\begin{notation}[Hypotheses]
\label{not:hypotheses-truth}
\textbf{Assumptions:} $(\mathsf{Obs},\preceq)$ is an $\omega$-cpo; $\mathcal{R}$ is monotone and $\omega$-continuous; boundary projections preserve order; equality layers respect joins.
\end{notation}

\begin{theorem}[Truth as Fixed Point in L/B/R Structure]
\label{thm:truth-fixed-point-lbr}
Let $(\mathsf{Obs},\preceq)$ be an $\omega$-cpo with L/B/R structure. Assume $\mathcal{R}:\mathsf{Obs}\to\mathsf{Obs}$ is monotone and $\omega$-continuous with respect to the simplified equality hierarchy $\equiv_\star, \equiv_B, \equiv_{\text{meta}}$. Then $\mathrm{lfp}(\mathcal{R})=\sup_n \mathcal{R}^n(\bot)$ exists (Kleene) and respects the L/B/R boundaries.
\end{theorem}

\begin{proof}[Proof Sketch]
The proof follows from Kleene's fixed point theorem, but now incorporates the L/B/R structure. Since $\mathcal{R}$ is monotone and $\omega$-continuous, the sequence $\{\mathcal{R}^n(\bot)\}_{n \geq 0}$ forms a chain in the $\omega$-cpo. The L/B/R boundaries are preserved under the RG flow because:
\begin{itemize}
\item Left boundary $L$: Truth evaluation via $\equiv_L$ equality
\item Bulk $B$: Computational dynamics via $\equiv_B$ equality  
\item Right boundary $R$: Truth evaluation via $\equiv_R$ equality
\end{itemize}
By Kleene's theorem, this chain has a least upper bound which is the least fixed point, and the L/B/R structure is preserved throughout the iteration.
\end{proof}

\begin{remark}[Connection to Implementation]
\label{rem:implementation-connection-lbr}
The RG operator iteration corresponds to iteration of the equality checking procedures in our implementation. The monotonicity condition is satisfied by the semiring operations in the M3/M2/M1 hierarchy, and the completeness follows from the $\omega$-cpo structure of our observable space. The L/B/R boundaries are implemented as:
\begin{itemize}
\item \texttt{M2\_pgc.rkt}: Left and right boundary DSLs
\item \texttt{M3\_rules.rkt}: Bulk computational dynamics
\item \texttt{M2\_cert.rkt}: Equality hierarchy implementation
\end{itemize}
\end{remark}

\subsection{Regularization as Deformation of L/B/R Structure}

Regularization deforms the L/B/R structure in a specific way, analogous to how temperature deforms crystal structures in condensed matter physics. For our HEP-TH audience, this deformation should feel familiar: just as we deform field theories by introducing regulators (like dimensional regularization), we deform the L/B/R structure by introducing computational regulators. See Table~\ref{tab:universal-domain-translation} for the single cross-domain dictionary used later.

\subsection{Equality Layers and Deformation}

The logic features a three-layer equality hierarchy (as in Definition~\ref{def:equality-hierarchy}): $\equiv_\star$ (reversible), $\equiv_B$ (bulk), $\equiv_{\text{meta}}$ (global), with hierarchy $\equiv_\star \subseteq \equiv_B \subseteq \equiv_{\text{meta}}$. Deformation promotes truth along the hierarchy: $\text{True}_\alpha = \lim_{\Lambda \to \infty} \mathcal{R}_\Lambda(\text{True}_{\alpha-1})$ where each level respects the appropriate equality notion.

\subsection{Spectral Gap and Computational Semantics}

The spectrum of the $\mathsf{Gen4}$ primitive has a natural symmetry-related gap between reversible computations (information preserving, $\equiv_\star$ equality) and irreversible computations (information destroying, $\equiv_{\text{meta}} \setminus \equiv_\star$), measured by the three-layer equality hierarchy (as in Definition~\ref{def:equality-hierarchy}). \textbf{Transfer-operator specifics:} See Section 12.

On scattering ports the reversible fragment aligns with forward‑channel monotonicity of $\sigma_{\alpha;N}$ under RG coarse‑graining (contractivity hypotheses of §5), giving an optical (c)‑like monotone.

\textbf{Physics interpretation:} This spectral gap corresponds to the separation between elastic (reversible) and inelastic (irreversible) scattering channels in quantum field theory.

\subsection{Truth as RG Fixed Point}

\begin{theorem}[Truth as RG Fixed Point in Three-Layer Equality Hierarchy]
\label{thm:truth-rg-fixed-point-lbr}
A statement $\phi$ is true if and only if it corresponds to a fixed point under RG flow within the L/B/R structure:
\[
\text{True}(\phi) \Leftrightarrow \lim_{\Lambda \to \infty} \mathsf{Gen4}(\llbracket\phi\rrbracket, \llbracket\bar{\phi}\rrbracket; \vec{q}, \Lambda) \text{ converges}
\]
where convergence of the RG flow corresponds to reversible computation (information preservation) and respects the simplified equality hierarchy (as in §7.7). The implicit functional arguments $z, \bar{z}$ are understood to be present (see Section~\ref{sec:computation-paradigms}).

\textbf{Entropy monotonicity:} $\mathrm{d}S/\mathrm{d}t\le 0$ under coarse-graining with equality iff reversible, pointing to Appendix C for metric/c-function formulas.
\end{theorem}

\subsection{Computational Semantics}

\begin{definition}[Computational Semantics in Three-Layer Equality Hierarchy]
\label{def:computational-semantics-lbr}
The computational semantics of our framework assigns meaning to logical statements through RG flow behaviour within the L/B/R structure and three-layer equality hierarchy:
\begin{itemize}
\item Truth: Converging RG flow (reversible computation) - corresponds to $\equiv_\star$ equality
\item Falsehood: Diverging RG flow (irreversible computation) - corresponds to $\equiv_{\text{meta}} \setminus \equiv_\star$
\item Undecidability: Marginal RG flow (undecidable computation) - corresponds to $\equiv_B \setminus \equiv_\star$
\end{itemize}
Each semantic notion respects the appropriate equality layer in the three-layer hierarchy.
\end{definition}

\subsection{Summary and Outlook}

This section has established truth as a fixed point under RG flow within the L/B/R structure and three-layer equality hierarchy (as in §7.7), providing the computational semantics for our framework. The key observations are:

\begin{enumerate}
\item Regularization deforms the L/B/R structure through RG flow
\item Complete three-layer equality hierarchy creates a hierarchy of truth predicates
\item L/B/R boundaries enable holographic renormalisation with $\mathsf{Gen4}$ as correlator
\item The spectral gap distinguishes reversible from irreversible computation via three-layer equality hierarchy (as in §7.7)
\item Truth corresponds to converging RG flow (fixed points) respecting all three equality layers
\item Computational semantics assigns meaning through RG flow behaviour within L/B/R structure and three-layer equality hierarchy (as in §7.7)
\end{enumerate}

The connection between truth and RG flow within the L/B/R structure and three-layer equality hierarchy provides a natural bridge between logic and physics. In the next section, we will develop the effective logic framework that unifies all computational paradigms through the MDE pyramid structure (Section~\ref{sec:effective-logic}), building on the truth semantics established here.

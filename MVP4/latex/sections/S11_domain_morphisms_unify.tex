\section{Domain Morphisms and Universal Invariants}
\label{sec:unified-theory}

Having established the complete framework spanning computation, logic, and physics through renormalization group flow within the L/B/R structure, we now examine the domain morphisms that connect our logical framework to various mathematical domains. These morphisms reveal universal invariants that provide consistency across different applications and offer machine-checkable routes to fundamental problems in mathematics and physics.

\paragraph{Domain recap.} Table~\ref{tab:domain-summary} summarises the contributions of each domain and the logic artefacts they instantiate. It is the final entry in our domain ledger and should assist the reader in navigating future extensions.

\begin{table}[h]
  \centering
  \begin{tabular}{@{}llll@{}}
    \toprule
    Domain & Imports from logic layer & Domain-specific structure & Exports back to logic \\
    \midrule
    Computation & $\mathsf{Gen4}$, L/B/R structure, equality hierarchy & Paradigm gradings $(q_1,q_2,q_3)$ & Normalised correlators, moduli space points \\
    Physics & RG operators, fixed-point theorems, $\equiv_\star$ equality & Green's functions, conformal blocks & Renormalised truth predicates, boundary maps \\
    LLMs & PGC evaluation, MDE hierarchy, $\equiv_{\text{meta}}$ equality & Stochastic dynamics, $(N,D,C,T)$ moduli & Training beta functions, empirical correlators \\
    Number theory/GRH & Transfer operators, spectral logic, $\equiv_{\text{loc}}$ equality & Hilbert-Polya operator, gaps & Conjectural spectral invariants \\
    \bottomrule
  \end{tabular}
  \caption{Domain summary highlighting how each application reuses the universal logic layer with L/B/R structure and equality hierarchy.}
  \label{tab:domain-summary}
\end{table}

\subsection{The Two-Observer Information Exchange Model in L/B/R Structure}

\subsection{The G6 Modal Convolution Framework}

The unified theory emerges through the domain morphism $\Psi^{G6}$ that translates modal enrichments of our logic into weighted convolution operators on a six-generator monoid $G_6$. This provides the analytic foundation for understanding language model phenomena and universal invariants.

\begin{definition}[G6 Modal Convolution Morphism $\Psi^{G6}$]
\label{def:g6-modal-morphism}
The morphism $\Psi^{G6}$ maps our modal Gen4 calculus to weighted convolutions on $G_6$:
\begin{itemize}
\item Source: Modal extension with operators $\Diamond$, $\Box$, and graded family $\Diamond_i$
\item Target: Monoid $G_6 = \langle g_1,\ldots,g_6 \rangle$ capturing elementary moves between latent states
\item Convolution algebra: $\mathscr{C}_{G_6} = (\mathcal{M}(G_6), +, *)$ with weighted operators
\item Kernel action: $K = \sum_i w_i \delta_{g_i}$ acts via convolution on functions/distributions
\end{itemize}
The morphism converts syntactic modal reasoning into statements about dynamical systems, providing analytic tools for studying universal properties of large language models.
\end{definition}

\begin{theorem}[LLM Interpretation via $\Psi^{G6}$]
\label{thm:llm-g6-interpretation}
Large language models can be understood as iterated application of modal operators through $\Psi^{G6}$:
\begin{itemize}
\item Each layer/head corresponds to a kernel $K_{layer}$ in the convolution algebra
\item Training adjusts weights $w_i$ to control invariants (loss, calibration metrics)
\item Scaling laws manifest as phase transitions in repeated convolution dynamics
\item Double-dip behavior corresponds to renormalization phenomena in convolution algebras
\end{itemize}
\subsection{The * Symmetry and Universal Invariants}

The $\equiv_\star$ equality (reversible equality) plays a central role in the unified theory, providing the symmetry that underlies all domain morphisms and universal invariants.

\begin{definition}[* Symmetry as Universal Invariant]
\label{def:star-symmetry-universal}
The $\equiv_\star$ equality provides a universal symmetry that:
\begin{itemize}
\item Preserves information under all domain morphisms ($\Phi^{HP}$, $\Psi^{cl}$, $\Psi^{G6}$)
\item Corresponds to unitary equivalence in operator algebras
\item Maps to reversible computation in complexity theory
\item Provides the kernel-cokernel split in spectral decompositions
\end{itemize}
This symmetry serves as the fundamental invariant across all domains, ensuring consistency of the unified framework.
\end{definition}

\begin{theorem}[Universal Invariants via * Symmetry]
\label{thm:universal-invariants-star}
The $\equiv_\star$ symmetry generates universal invariants across all domain morphisms:
\begin{itemize}
\item Under $\Phi^{HP}$: Maps to unitary equivalence preserving zeta-function functional equations
\item Under $\Psi^{cl}$: Maps to P-subspace invariance in complexity spectral algebras
\item Under $\Psi^{G6}$: Maps to unitary convolution kernels preserving total mass and entropy measures
\end{itemize}
These invariants provide machine-checkable consistency conditions across all domains.
\end{theorem}

\begin{theorem}[Effective-Theory Payoffs via * Symmetry]
\label{thm:effective-theory-payoffs}
The $\equiv_\star$ symmetry provides effective-theory insights:
\begin{itemize}
\item Universal invariants: Quantities preserved under convolution become invariants of the modal calculus
\item Stability: Spectral norms of convolution operators diagnose training stability
\item Scaling phenomena: Double-dip and scaling laws arise from repeated convolution statistics
\item Learning vs Maxwell agents: Information concentration corresponds to kernel sharpening with compensation elsewhere
\end{itemize}
The * symmetry provides the mathematical foundation for understanding universal properties of large language models.
\end{theorem}

\begin{definition}[Two-Observer Information Exchange in L/B/R Structure]
\label{def:two-observer-exchange-lbr}
The computational process is modeled as information exchange between two observers within the L/B/R structure:
\begin{itemize}
\item Observer A (Left Boundary $L$): Encodes computational state $\phi$ with information content $I_A(\phi)$ respecting $\equiv_L$ equality
\item Observer B (Right Boundary $R$): Decodes computational state $\psi$ with information content $I_B(\psi)$ respecting $\equiv_R$ equality
\item Information Channel (Bulk $B$): The $\mathsf{Gen4}$ primitive mediates the exchange with channel capacity $C(\mathsf{Gen4})$ respecting $\equiv_B$ equality
\end{itemize}
The mutual information between observers is:
\[
I(A;B) = I_A(\phi) + I_B(\psi) - I_{AB}(\phi,\psi)
\]
where $I_{AB}(\phi,\psi)$ is the joint information content, and all information measures respect the equality hierarchy $\equiv_L, \equiv_B, \equiv_R, \equiv_{\text{loc}}, \equiv_{\text{meta}}, \equiv_\star$.
\end{definition}

This model naturally connects to the holographic renormalization structure from Section~\ref{sec:boundary-maps}, where the L/B/R boundaries correspond to the input and output of the $\mathsf{Gen4}$ primitive, and the information exchange corresponds to the computational process itself.

\subsection{Fisher Information Metric and c-Function}

The Fisher information metric provides the natural geometric structure for understanding information flow in our computational framework within the L/B/R structure.

\begin{definition}[Fisher Information Metric for $\mathsf{Gen4}$]
\label{def:fisher-metric-g6}
The Fisher information metric for our $\mathsf{Gen4}$ primitive is:
\[
g_{ij}(\vec{q}) = \mathbb{E}\left[\frac{\partial \log \mathsf{Gen4}}{\partial q_i} \frac{\partial \log \mathsf{Gen4}}{\partial q_j}\right]
\]
where the expectation is taken over the computational state distribution, and all derivatives respect the equality hierarchy $\equiv_L, \equiv_B, \equiv_R, \equiv_{\text{loc}}, \equiv_{\text{meta}}, \equiv_\star$.
\end{definition}

\begin{theorem}[c-Function from Fisher Information via $\mathsf{Gen4}$]
\label{thm:c-function-g6}
The c-function emerges naturally from the Fisher information metric:
\[
c(\Lambda) = \frac{1}{2} \text{Tr}(g_{ij}(\vec{q}(\Lambda)))
\]
where $\vec{q}(\Lambda)$ are the running parameters under RG flow. The c-function satisfies the c-theorem:
\[
\frac{dc}{d\Lambda} \leq 0
\]
with equality only at RG fixed points where $\mathsf{Gen4}$ respects $\equiv_\star$ equality.
\end{theorem}

The c-function provides a natural measure of information content that decreases monotonically under RG flow, corresponding to the coarse-graining of information in the computational process.

\subsection{a-Function and Anomaly Coefficients}

The a-function provides another fundamental information-theoretic measure connected to anomaly coefficients in quantum field theory.

\begin{definition}[a-Function]
\label{def:a-function}
The a-function is defined through the anomaly coefficients of our computational system:
\[
a(\Lambda) = \frac{1}{24\pi^2} \left[ \text{Tr}(R^2) - \frac{1}{4}\text{Tr}(R \wedge R) \right]
\]
where $R$ is the curvature tensor of the Fisher information metric $g_{ij}$.
\end{definition}

\begin{theorem}[a-Function Monotonicity]
\label{thm:a-function-monotonicity}
The a-function satisfies the a-theorem:
\[
\frac{da}{d\Lambda} \leq 0
\]
with equality only at conformal fixed points. This provides a fundamental constraint on information flow in computational systems.
\end{theorem}

The a-function and c-function together provide complementary measures of information content that govern the flow of information in our computational framework.

\subsection{Multiple Entropy Types and Their Roles}

Our unified framework naturally incorporates multiple types of entropy, each playing a distinct role in the computational process.

\subsubsection{Thermodynamic Entropy (Fundamental)}

\begin{definition}[Thermodynamic Entropy]
\label{def:thermodynamic-entropy}
The fundamental thermodynamic entropy of the computational system is:
\[
S_{\text{thermo}}(\Lambda) = k_B \log \Omega(\Lambda)
\]
where $\Omega(\Lambda)$ is the number of microstates accessible at scale $\Lambda$, and $k_B$ is Boltzmann's constant.
\end{definition}

This entropy represents the fundamental information content of the computational system and provides the thermodynamic foundation for all other entropy measures.

\subsubsection{Shannon Entropy}

\begin{definition}[Shannon Entropy]
\label{def:shannon-entropy}
The Shannon entropy measures the information content of the computational state distribution:
\[
S_{\text{Shannon}}(\Lambda) = -\sum_{n,m} p_{n,m}(\Lambda) \log p_{n,m}(\Lambda)
\]
where $p_{n,m}(\Lambda) = \frac{|\mathcal{Z}_{n,m}(\vec{q})|}{\sum_{n',m'} |\mathcal{Z}_{n',m'}(\vec{q})|}$.
\end{definition}

This is the entropy measure we introduced in Section~\ref{sec:rg-flow} and provides the connection to information theory.

\subsubsection{Von Neumann Entropy}

\begin{definition}[Von Neumann Entropy]
\label{def:von-neumann-entropy}
The von Neumann entropy measures the quantum information content:
\[
S_{\text{vN}}(\Lambda) = -\text{Tr}(\rho(\Lambda) \log \rho(\Lambda))
\]
where $\rho(\Lambda)$ is the density matrix of the computational state at scale $\Lambda$.
\end{definition}

\subsubsection{Rényi Entropy}

\begin{definition}[Rényi Entropy]
\label{def:renyi-entropy}
The Rényi entropy provides a family of entropy measures:
\[
S_{\alpha}(\Lambda) = \frac{1}{1-\alpha} \log \sum_{n,m} p_{n,m}(\Lambda)^\alpha
\]
where $\alpha \geq 0$ is the Rényi parameter. Special cases include:
\begin{itemize}
\item $\alpha \to 1$: Shannon entropy
\item $\alpha = 0$: Hartley entropy (logarithm of support size)
\item $\alpha = 2$: Collision entropy
\item $\alpha \to \infty$: Min-entropy
\end{itemize}
\end{definition}

\subsection{Entropy Relationships and Information Flow}

\begin{theorem}[Entropy Hierarchy]
\label{thm:entropy-hierarchy}
The different entropy measures satisfy the hierarchy:
\[
S_{\text{thermo}} \geq S_{\text{vN}} \geq S_{\text{Shannon}} \geq S_{\alpha} \geq S_{\infty}
\]
with equality only at RG fixed points where all entropy measures coincide.
\end{theorem}

\begin{theorem}[Information Flow Conservation]
\label{thm:info-flow-conservation}
Under RG flow, the total information content is conserved:
\[
\frac{d}{d\Lambda} \left[ S_{\text{thermo}} + I(A;B) + c(\Lambda) + a(\Lambda) \right] = 0
\]
This provides a fundamental conservation law for information in computational systems.
\end{theorem}

\subsection{Unified Information-Theoretic Truth Criteria via L/B/R Structure}

The multiple entropy measures provide a unified framework for understanding truth as an information-theoretic concept within the L/B/R structure.

\begin{definition}[Information-Theoretic Truth via Equality Hierarchy]
\label{def:info-truth-lbr}
A computational statement $\phi$ is true if and only if:
\begin{enumerate}
\item Thermodynamic condition: $S_{\text{thermo}}(\phi) = S_{\text{thermo}}(\text{vacuum})$ (respects $\equiv_\star$ equality)
\item Information condition: $I(A;B) = \max$ (maximum mutual information respecting $\equiv_{\text{meta}}$ equality)
\item Entropy condition: $S_{\text{Shannon}}(\phi) = 0$ (no information loss respecting $\equiv_{\text{loc}}$ equality)
\item Flow condition: $\frac{dc}{d\Lambda} = \frac{da}{d\Lambda} = 0$ (RG fixed point respecting $\equiv_B$ equality)
\item Boundary condition: $\phi$ respects both $\equiv_L$ and $\equiv_R$ equalities
\end{enumerate}
\end{definition}

This definition unifies the thermodynamic, information-theoretic, and RG flow perspectives on truth within the L/B/R structure and equality hierarchy.

\subsection{The Mass Gap: A Key Insight from the L/B/R Construction}

The mass gap provides one important insight from our L/B/R construction—it offers a spectral perspective on computational behavior that connects to established physics concepts.

\begin{definition}[Mass Gap in $\mathsf{Gen4}$ Spectrum]
\label{def:mass-gap-spectrum-g6}
The mass gap $\Delta m$ in the $\mathsf{Gen4}$ primitive spectrum is:
\[
\Delta m = \lambda_{\text{ground}} - \lambda_{\text{first excited}}
\]
where $\lambda_{\text{ground}}$ is the ground state eigenvalue (reversible computation respecting $\equiv_\star$ equality) and $\lambda_{\text{first excited}}$ is the first excited state eigenvalue (irreversible computation respecting $\equiv_{\text{meta}} \setminus \equiv_\star$).
\end{definition}

This provides a spectral perspective on computational behavior within the L/B/R structure, where different regions of the spectrum correspond to different computational properties and equality layers.

\subsubsection{Physical Interpretation of the Mass Gap in L/B/R Structure}

The mass gap connects our computational framework to established concepts in quantum field theory through the L/B/R boundaries:

\begin{definition}[Mass Gap as Information Stability via Equality Hierarchy]
\label{def:mass-gap-stability-lbr}
The mass gap $\Delta m$ provides a measure of information stability in the computational system within the L/B/R structure:
\begin{itemize}
\item $\Delta m > 0$ (Mass Gap): Information is stable—small perturbations cannot destroy the computational state (respects $\equiv_\star$ equality)
\item $\Delta m = 0$ (No Mass Gap): Information is marginally stable—arbitrarily small perturbations can destroy the computational state (respects $\equiv_{\text{loc}}$ equality)
\item $\Delta m < 0$ (Negative Mass): Information is unstable—the computational state spontaneously decays (respects $\equiv_{\text{meta}} \setminus \equiv_\star$)
\end{itemize}
\end{definition}

This offers one perspective on why some computations preserve information while others destroy it, within our multilayered truth framework.

\subsubsection{Connection to Yang-Mills Mass Gap Problem}

Our construction offers a computational perspective on the Yang-Mills mass gap problem:

\begin{remark}[Computational Perspective on Yang-Mills Mass Gap]
\label{rem:computational-yang-mills}
The Yang-Mills mass gap problem may be related to the mass gap in our logic transformer spectrum. This connection remains speculative and requires further investigation to establish whether:
\[
\text{Yang-Mills mass gap exists} \Leftrightarrow \Delta m_{\text{logic}} > 0
\]
where $\Delta m_{\text{logic}}$ is the mass gap in our computational framework.
\end{remark}

This represents one potential connection between our computational framework and established physics problems.

\subsubsection{Mass Gap and Information Flow}

The mass gap provides one factor influencing information flow in our two-observer model:

\begin{proposition}[Mass Gap Influence on Information Flow]
\label{prop:mass-gap-info-flow}
The mass gap may influence the rate of information exchange between observers:
\[
\frac{dI(A;B)}{dt} \propto -\Delta m \cdot I(A;B)
\]
This suggests:
\begin{itemize}
\item Large mass gap: Information exchange tends to be more stable
\item Zero mass gap: Information exchange may be marginal
\item Negative mass gap: Information exchange may be unstable
\end{itemize}
\end{proposition}

This represents one aspect of the multilayered truth structure, where multiple factors contribute to computational behavior.

\subsection{The Boundary Problem and Observer Physics}

The mass gap construction naturally leads to the boundary problem: if computation fundamentally requires boundaries (input and output observers), then we must address the question of the universe's boundary.

\begin{conjecture}[Universal Boundary Problem]
\label{conj:universal-boundary}
The universe itself can be understood as a computational system requiring boundaries. The fundamental questions are:
\begin{itemize}
\item What is the boundary of the universe? If the universe is a computational system, what serves as its input and output boundaries?
\item Who observes the universe? If there are no internal observers, what external observer defines the universe's computational semantics?
\item Self-referential computation: Can the universe observe itself, creating a self-referential computational loop?
\end{itemize}
\end{conjecture}

The mass gap provides the mechanism by which the universe can maintain stable information exchange with its boundary observers, making this computational model of reality mathematically consistent.

\subsection{On the Epistemic Status of the Framework}

The epistemic status of our framework rests on three conditional claims. If the framework truly defines a logic (claim 1) and is consistent (claim 2), as well as our implementation of it (claim 3), then it is plausible that the domain map to the respective domain will align. This alignment is almost guaranteed by the observation that our axioms likely align with those of ZFC, providing a solid mathematical foundation.

The remaining question is whether "the words fit"—that is, whether the semantic interpretations of our formal structures correspond to the intended computational, physical, and logical phenomena. This is fundamentally not a question of logic but of semantics, requiring empirical validation and domain-specific interpretation.

At minimum, the present framework sketches a rather direct and machine-checked route to three of the biggest problems in mathematics and mathematical physics: the Yang-Mills mass gap problem, the Riemann hypothesis (through the Hilbert-Polya connection), and the P vs NP problem (through the spectral gap classification). Whether these connections prove fruitful depends on the semantic alignment between our formal structures and the target domains.

\subsection{Mathematical Structure as Fundamental Reality}

The systematic applications of logic throughout this paper provide evidence for fundamental mathematical structure underlying physical reality. The information-theoretic framework reveals that:

\begin{enumerate}
\item Computation is information exchange: All computational processes reduce to information exchange between observers
\item Truth is information preservation: Truth corresponds to maximum information preservation in the exchange
\item Physics is computational semantics: Physical laws emerge as the semantics of universal computation
\item Mathematics is the boundary condition: Mathematical structure provides the boundary conditions that make universal computation well-defined
\end{enumerate}

\subsection{Summary and Philosophical Implications}

This unified theory of computation, information, and truth reveals that:

\begin{enumerate}
\item Domain morphisms provide rigorous bridges between logic and analytic domains: $\Phi^{HP}$ connects to Hilbert-Polya operator algebras, $\Psi^{cl}$ to complexity spectral algebras, and $\Psi^{G6}$ to convolution algebras
\item The $\equiv_\star$ symmetry serves as the universal invariant across all domains, ensuring consistency and providing machine-checkable routes to fundamental problems
\item Fisher information metric provides the geometric structure for information flow
\item c-function and a-function provide fundamental constraints on information content
\item Multiple entropy types each capture different aspects of information in computational systems
\item Two-observer model provides the fundamental framework for understanding computation as information exchange
\item G6 modal convolution framework explains LLM training dynamics and scaling laws through analytic tools
\item Boundary problem emerges naturally from the requirement that computational systems need observers
\item Mathematical structure appears as the fundamental reality underlying physical and computational phenomena
\end{enumerate}

The framework supports the view that logic systems are fundamentally "open" systems requiring boundaries for definition. When translated to observer-style physics, this leads to familiar discussions about quantum mechanics interpretation, but with the computational twist that any system fundamentally needs boundaries—raising the profound question of what serves as the boundary of the universe itself.

The author interprets the systematic applications of logic in this paper as evidence of fundamental mathematical structure, where computation, information, and truth are unified through the renormalization group flow of information exchange between observers.

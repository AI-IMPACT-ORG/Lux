\section{Effective Logic as MDE-Pyramid of Logics}
\label{sec:effective-logic}

Having established truth as a fixed point under RG flow in Section~\ref{sec:truth-fixed-point}, we now turn to the effective logic framework that provides a hierarchy of logics unifying all computational paradigms. This is the MDE (Model-Driven Engineering) pyramid structure that enables the unification.

\subsection{Convolution of Formal Systems and Effective Theory}

\begin{definition}[Convolution of Formal Systems]
\label{def:convolution-formal}
We use a \textbf{monoidal product} $\star$ on logics with conservative projections; $\mathcal{L}_1\star\mathcal{L}_2$ has signature $\Sigma_1\sqcup\Sigma_2$ and non-interfering coupling rules. Large Language Models can be understood as convolutions of basic formal systems. Given a base formal system $\mathcal{L}_0$, the convolution creates extensions:
\[
\mathcal{L}_n = \mathcal{L}_0 \star \mathcal{L}_0 \star \cdots \star \mathcal{L}_0 \quad \text{(n times)}
\]
\end{definition}

\begin{definition}[Effective Theory of Computation]
\label{def:effective-computation}
$\mathcal{L}_{\text{eff}}$ is a \textbf{conservative extension} of $\mathcal{L}_0$ controlled by couplings $\{g_i\}$ (operators $\{\mathcal{O}_i\}$ enabled in the extension).
\end{definition}

\begin{theorem}[LLMs as Formal Language Models]
\label{thm:llm-formal}
Large Language Models are (controllable extensions of) convolutions of a basic class of formal language models, where the base formal system defines a formal language $\mathcal{L}_0$, convolution creates extensions $\mathcal{L}_n$ with additional structure, training learns the coupling constants $g_i$ that determine the effective theory, and the extensions are controllable through the coupling constants.
\end{theorem}

\subsection{MDE Pyramid Structure}

\begin{definition}[MDE Pyramid]
\label{def:mde-pyramid}
The MDE pyramid provides a hierarchical organization that unifies all computational paradigms:
\begin{itemize}
\item \textbf{Level 0}: Basic logical primitives (6-ary connective $\mathsf{G}_6$)
\item \textbf{Level 1}: Computational paradigms (Turing $(1,0,0)$, Church $(0,1,0)$, Feynman $(0,0,1)$)
\item \textbf{Level 2}: Domain models (computation, physics, mathematics)
\item \textbf{Level 3}: Applications (LLMs, complexity, number theory)
\end{itemize}
\end{definition}

This hierarchy mirrors effective field theories at different energy scales in QFT, with effective logics at different abstraction levels. The effective action at each level can be written as:
\[
S_{\text{eff}}^{(n)} = S_0 + \sum_{i=1}^n g_i^{(n)} \mathcal{O}_i^{(n)} + \text{higher order terms}
\]
where $g_i^{(n)}$ are the coupling constants at level $n$ and $\mathcal{O}_i^{(n)}$ are the operators.

\begin{definition}[Convolution]
\label{def:convolution}
Given logics $\mathcal{L}_1,\mathcal{L}_2$ over $\Sigma_1,\Sigma_2$, their convolution
$\mathcal{L}_1\star\mathcal{L}_2$ has signature $\Sigma_1\sqcup\Sigma_2$ and proof rules
generated by $\vdash_1\cup\vdash_2\cup\mathcal{C}$ where coupling rules $\mathcal{C}$ are
noninterfering on each base fragment. Projections to each factor are conservative.
\end{definition}

\subsection{Hierarchy of Logics}

\begin{definition}[Hierarchy of Logics]
\label{def:hierarchy-logics}
The hierarchy: $\mathcal{L}_0 \subseteq \mathcal{L}_1 \subseteq \mathcal{L}_2 \subseteq \mathcal{L}_3$ where $\mathcal{L}_0$ is basic logic, $\mathcal{L}_1$ computational logic, $\mathcal{L}_2$ domain logic, and $\mathcal{L}_3$ application logic.
\end{definition}

\subsection{Inter-Level Mappings}

\begin{definition}[Inter-Level Mappings]
\label{def:inter-level-mappings}
The mappings between levels are:
\begin{itemize}
\item \textbf{Level 0 → Level 1}: Logic transformer generates computational paradigms
\item \textbf{Level 1 → Level 2}: Paradigms generate domain models
\item \textbf{Level 2 → Level 3}: Domain models generate applications
\end{itemize}
\end{definition}

\subsection{Summary and Outlook}

This section has established the effective logic framework through the MDE pyramid structure. The key observations are:

\begin{enumerate}
\item Convolution of formal systems creates extensions with additional coupling constants
\item Effective theory of computation emerges from the convolution structure
\item LLMs are controllable extensions of convolutions of formal language models
\item MDE pyramid provides hierarchical organization of computational paradigms
\item Hierarchy of logics creates mathematical structure across levels
\item Inter-level mappings connect different levels of abstraction
\end{enumerate}

The effective logic framework provides a unified structure for understanding how different computational paradigms emerge from the same underlying logical foundation. In the next section, we will show how this framework reproduces and generalizes known theorems from logic, physics, and computation.
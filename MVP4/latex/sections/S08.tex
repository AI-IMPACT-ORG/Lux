\section{Consistency, Compactness - Relation to Known Theorems}
\label{sec:consistency}

Having established the interacting positive logic framework in Section~\ref{sec:formal-systems}, we now demonstrate how our approach reproduces and generalizes known theorems from logic, physics, and computation. This section builds on the $\mathsf{Gen4}$ primitive from Section~\ref{sec:computation-paradigms} and the simplified equality hierarchy. In the physics domain, these consistency theorems correspond to unitarity, crossing symmetry, and cluster decomposition properties of S-matrix elements.

This provides crucial validation that our framework is not only novel but also mathematically sound and consistent with established results. Each theorem translates across domains through systematic translation maps. The consistency results directly validate the Green's functions hierarchy through convergence properties and fixed-point behaviour, implementing the five "truth theorems" from the CLASS specification: bulk = two boundaries, umbral-renormalisation commutativity, Church$\leftrightarrow$Turing equivalence, EOR (each object represented once), and logic-$\zeta$ critical-line equivalence.

\subsection{Seven Core Theorems}

For the canonical statement of the core theorems, see Appendix~\ref{app:technical-derivations}. This section applies those results to cross-checks against classical theorems and physics mappings; detailed proofs remain in Appendix~\ref{app:technical-derivations}.

\subsection{Internal complexity via the equality hierarchy (mechanised evidence)}

\begin{definition}[Blum complexity measure \cite{blum1967}]
A partial $\Phi(P,x)$ is a Blum measure iff
(i) $\Phi(P,x)$ defined $\Leftrightarrow$ $P(x)$ halts;
(ii) $\{(P,x,t)\mid \Phi(P,x)\le t\}$ is decidable.
\end{definition}

\begin{notation}[Mechanised Evidence]
\label{not:mechanised-evidence}
\textbf{Mechanised Evidence:} Mechanised runs support that $\Phi_{\mathsf{Gen4}}$ satisfies Blum's axioms on the base fragment, assuming bounded-sum decidability; the extension to full semantics remains conditional and is stated as a conjectural programmeme.

\textbf{Assumptions:} Bounded-sum predicate $\sum_{n+m\le T}\mathcal{Z}_{n,m}\ge\theta(x)$ is decidable; $\mathcal{Z}_{n,m}\ge0$; comparisons computed modulo qmask; proper model-theoretic semantics for $\mathsf{Gen4}$ convergence.
\end{notation}

\begin{notation}[Conjectural Interpretation]
\label{not:conjectural-interpretation}
\begin{conjecture}[Internal measure via Equality Hierarchy (Conjectural Interpretation)]
\label{conj:internal-measure}
Let $\Phi_{\mathsf{Gen4}}(x):=\min\{T\mid \sum_{n+m\le T}\mathcal{Z}_{n,m}\ge \theta(x)\}$ for a fixed threshold $\theta$.

\textbf{Conjectural Interpretation:} Under explicit decidability and semantics assumptions, the complexity measure may satisfy Blum axioms within the L/B/R structure:
\begin{align}
\text{(i) Domain condition: } &\Phi_{\mathsf{Gen4}}(x) \text{ defined } \Leftrightarrow \lim_{\Lambda \to \infty} \mathsf{Gen4}(\vec{q}, \Lambda) \text{ converges under } \equiv_{\text{meta}} \\
\text{(ii) Decidability: } &\{(G,x,t) \mid \Phi_{\mathsf{Gen4}}(x) \leq t\} \text{ is decidable given our observable and equality hierarchy}
\end{align}

\textbf{Status:} This remains a conjectural interpretation until proper model-theoretic semantics are established for $\mathsf{Gen4}$ convergence and the bounded-sum predicate decidability is proven rather than assumed.

\textbf{Domain Translation}: See Table~\ref{tab:universal-domain-translation} for domain-specific complexity interpretations.
\end{conjecture}
\end{notation}

\begin{notation}[Interpretive Heuristics]
\label{not:interpretive-heuristics}
\begin{conjecture}[P vs NP via Transfer-Operator Spectrum (Interpretive Heuristics)]
\label{conj:p-vs-np-spectrum}
Within our framework, P vs NP may reduce to a topological classification of the transfer-operator spectrum. Specifically:
\begin{itemize}
\item P problems: Correspond to eigenvalues in the kernel of the transfer operator (reversible computations)
\item NP problems: Correspond to eigenvalues in the co-kernel of the transfer operator (irreversible computations)
\item P = NP: Equivalent to the spectral gap between kernel and co-kernel being trivial
\end{itemize}
This conjecture is formulated within our specific computational framework and does not constitute a general complexity-theoretic result. Cross-reference to §10.2's Conjectural Interpretation.
\end{conjecture}
\end{notation}

\subsection{Generalised Rice Theorem for $\mathsf{Gen4}$ Generating Functions}

\begin{notation}[Speculative Programme]
\label{not:speculative-programme}
\begin{conjecture}[Generalised Rice via Equality Hierarchy (Programme)]
\label{conj:generalised-rice}
\textbf{Assumptions:} $\mathcal{Z}_{n,m}\ge0$; equality hierarchy $\equiv_\star, \equiv_B, \equiv_{\text{meta}}$ defines admissible semantic properties; comparisons computed modulo qmask; effective enumeration of computational objects; proper reduction from r.e. index sets.

\textbf{Programme:} Under explicit recursion-theoretic assumptions, every non-trivial semantic property of the renormalised $\mathsf{Gen4}$ generating function $\mathsf{Gen4}^{\text{ren}}(z,\bar{z};\vec q,\Lambda)$ that is invariant under the simplified equality hierarchy $\equiv_\star, \equiv_B, \equiv_{\text{meta}}$ may be undecidable.

\textbf{Status:} This remains a programmeme until proper recursion-theoretic underpinnings are established, including effective enumeration of computational objects and explicit reduction from recursively enumerable index sets. The current formulation borrows Rice's surface shape without the necessary recursion-theoretic foundations.
\end{conjecture}

\begin{corollary}[Consequences (Conditional)]
If the programmeme above is realised, it would imply that:
\begin{itemize}
  \item Determining whether a renormalised observable stabilises (i.e.\ whether a given RG scheme reaches a fixed point) may be undecidable in general.
  \item Any attempt to classify LLM training dynamics via a semantic predicate on $\mathcal{G}_{\text{LLM}}$ may inherit the same undecidability barrier.
  \item Number-theoretic instantiations (Section~\ref{sec:spectral-gap}) may not be able to algorithmically decide spectral gap existence once the predicate depends on the renormalised correlators.
\end{itemize}
\end{corollary}
\end{notation}

\subsection{Hilbert–Pólya Operator and Zeta-Function Interpretation}

\begin{notation}[Speculative Programme]
\label{not:speculative-programme-hilbert}
\begin{conjecture}[Hilbert–Pólya Connection]
\label{def:hilbert-polya}
The logic transformer may provide a Hilbert–Pólya operator $\mathcal{H}$ where eigenvalues correspond to Riemann zeta function zeros. This remains speculative (see Section~\ref{sec:spectral-gap} for concrete transfer operator).
\end{conjecture}

\begin{conjecture}[Zeta-Function Interpretation]
\label{conj:zeta-interpretation}
The natural heat-kernel regulator may provide a zeta-function interpretation:
\[
\zeta_{\mathcal{H}}(s) = \text{Tr}(\mathcal{H}^{-s}) = \sum_{\lambda} \lambda^{-s}
\]
where:
\begin{itemize}
\item The zeros of $\zeta_{\mathcal{H}}(s)$ may correspond to eigenvalues of $\mathcal{H}$
\item The Riemann hypothesis may be equivalent to the spectral gap being non-trivial
\item The logic transformer spectrum may determine the distribution of zeros
\end{itemize}
\textbf{Note:} This interpretation is conjectural and requires additional assumptions about the spectral structure.
\end{conjecture}

\begin{conjecture}[Hilbert–Pólya Scenario]
\label{conj:hilbert-polya-scenario}
The construction may conform to the Hilbert–Pólya scenario:
\begin{itemize}
\item The logic transformer may provide the Hilbert–Pólya operator
\item The eigenvalues may correspond to zeros of the zeta function
\item The spectral gap may determine the distribution of zeros
\item The Riemann hypothesis may follow from the non-triviality of the spectral gap
\end{itemize}
\textbf{Note:} This scenario is conjectural and does not claim resolution of the Riemann hypothesis.
\end{conjecture}
\end{notation}

\subsection{Mass-Gap Theorem Connection}

\begin{remark}[Mass-Gap Theorem]
\label{rem:mass-gap}
The spectral gap in the logic transformer spectrum is directly connected to the mass-gap theorem of quantum field theory:
\begin{itemize}
\item Mass gap: The difference between the ground state and first excited state
\item Spectral gap: The difference between kernel and co-kernel eigenvalues
\item Both gaps measure the stability of the respective systems
\end{itemize}
\end{remark}

\subsection{Domain Maps and Generality}

\begin{theorem}[Domain Map Generality]
\label{thm:domain-generality}
The results are general for any domain map - if the logic checks out. Specifically:
\begin{itemize}
\item Any domain map from computation to a mathematical structure preserves the complexity classification
\item The Blum axioms hold in any domain where the generating function structure is well-defined
\item The spectral gap classification applies universally across domains
\end{itemize}
\end{theorem}

The consistency results provide validation that our framework is not just novel, but also correct. For boundary maps and holographic renormalisation, see Section~\ref{sec:boundary-maps}.

For programmeme-level discussion of possible Hilbert--P\'olya operators and spectral mappings, see the Outlook (Section~\ref{sec:outlook}); no claims are used in proofs here.

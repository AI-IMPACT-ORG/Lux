\section{Applications to Number Theory and Computational Complexity}
\label{sec:spectral-gap}

Having established the domain morphisms framework in Section~\ref{sec:unified-theory}, we now examine specific applications to number theory and computational complexity. This section builds on the $\mathsf{Gen4}$ primitive from Section~\ref{sec:computation-paradigms}, the simplified equality hierarchy from Section~\ref{sec:formal-systems}, and the universal invariants from Section~\ref{sec:unified-theory}. In the physics domain, these applications correspond to spectral properties of quantum field theories and their connection to critical phenomena.

We demonstrate how the $\Phi^{HP}$ morphism connects our $\mathsf{Gen4}$ logic to Hilbert–Pólya operator algebras \cite{connes1997,berry1999,montgomery1973,odlyzko1987} for studying the Riemann hypothesis, and how the $\Psi^{cl}$ morphism provides insights into P vs NP through complexity spectral algebras. These applications illustrate the power of the domain morphism approach for connecting logical reasoning to fundamental mathematical problems. The spectral applications directly implement the Green's functions hierarchy through transfer operators \cite{ruelle1978,ruelle1989,mayer1991} and spectral gap analysis, with the Fisher-critical line providing the connection to the logic-$\zeta$ critical-line equivalence from the CLASS specification.

\subsection{Reversibility Constraint RC† and Spectral Applications}

The reversibility constraint RC† introduced in Section~\ref{sec:computation-paradigms} determines the structure of spectral applications in our system. This fundamental constraint distinguishes between two regimes:

\begin{definition}[Reversibility Constraint RC† on Spectral Applications]
\label{def:reversibility-constraint-spectral}
The reversibility constraint RC† affects spectral applications as follows:

\textbf{With Reversibility Constraint RC† (2D Case)}: Spectral applications respect dagger symmetry, leading to reversible spectral transformations with no information loss. This corresponds to classical deterministic computation where spectral properties are preserved under all operations.

\textbf{Without Reversibility Constraint RC† (4D Case)}: Spectral applications allow irreversible computation with information loss. The system reveals the full 4D structure with six variables enabling linearization of the complete structure. This corresponds to our novel framework where spectral applications reveal universal invariants.

The key insight is that the 4D case is not fundamentally different from the 2D case—it simply has more variables that enable linearization of the complete structure. See the canonical ledger Table~\ref{tab:universal-domain-translation} for the single cross-domain spectral dictionary.
\end{definition}

\subsection{Hilbert–Pólya Operator and Zeta-Function Interpretation}

The connection to the Hilbert–Pólya scenario emerges naturally through the domain morphism $\Phi^{HP}$ from our $\mathsf{Gen4}$ logic to Hilbert–Pólya operator algebras. This morphism provides the rigorous foundation for connecting logical axioms to analytic theorems about zeta functions.

\begin{definition}[Hilbert–Pólya Domain Morphism $\Phi^{HP}$]
\label{def:hilbert-polya-morphism}
The morphism $\Phi^{HP}$ maps our $\mathsf{Gen4}$ logic to a Hilbert–Pólya operator algebra $\mathcal{A}_K$:
\begin{itemize}
\item Target algebra: $\mathcal{A}_K = (\text{bounded operators on } \mathcal{H}_K) \times U(1)$
\item Hilbert space: $\mathcal{H}_K = L^2(K_{\mathbb{A}}^\times / K^\times, \mathrm{d}\mu)$ for number field $K$
\item Self-adjoint operator: $H_K$ with $\operatorname{Spec}(H_K) = \{\gamma_j\}$ (imaginary parts of nontrivial zeros of $\zeta_K$)
\item Completed zeta kernel: $\Xi_K(s) = \pi^{-ns/2} \Gamma_{\mathbb{R}}(s)^{r_1} \Gamma_{\mathbb{C}}(s)^{r_2} |d_K|^{s/2} \zeta_K(s)$
\end{itemize}
The morphism preserves logical axioms as analytic theorems, with each $\mathsf{Gen4}$ slot mapping to specific zeta-function evaluations. Cross-domain readings (gauge/presentation choices and their interpretations) are consolidated in Table~\ref{tab:universal-domain-translation}.
\end{definition}

\begin{definition}[Transfer operator and gap in L/B/R Structure]
\label{def:transfer-operator-lbr}
The transfer operator $\mathsf{T}_\Lambda$ acts on $\mathcal{H}=\ell^2(\mathbb{N}^2,\mu)$ within the L/B/R structure, corresponding to the Hilbert–Pólya operator $H_K$ under the morphism $\Phi^{HP}$. We assume $\mathsf{T}_\Lambda$ is positive and bounded with simple top eigenvalue $1$ (Perron–Frobenius regime). The spectral gap is:
\[
\gamma(\Lambda):=1-\sup\{|\lambda|:\lambda\in\mathrm{Spec}(\mathsf{T}_\Lambda)\setminus\{1\}\}
\]
This gap respects the simplified equality hierarchy $\equiv_\star, \equiv_B, \equiv_{\text{meta}}$, with all comparisons computed modulo the active quotient mask $qmask \subseteq \{\text{phase}, \text{scale}\}$ (default $\{\text{phase}\}$), and corresponds to the gap in the zeta-function zeros under $\Phi^{HP}$.
\end{definition}

\begin{theorem}[Exponential mixing via Equality Hierarchy]
\label{thm:mixing-convergence-lbr}
If $\gamma(\Lambda)\ge\gamma_0>0$, then for $f$ orthogonal to the top eigenspace, $\|\mathsf{T}_\Lambda^k f - \Pi f\|\le C e^{-\gamma_0 k}\|f\|$,
so RG iterates converge exponentially to the fixed point respecting $\equiv_\star$ equality (reversible computation).
\end{theorem}

\begin{conjecture}[Hilbert–Pólya Connection via $\Phi^{HP}$ Morphism (Conditional Equivalences)]
\label{conj:hilbert-polya-connection}
Under the domain morphism $\Phi^{HP}$ and standard regularity hypotheses, the transfer operator $\mathsf{T}_\Lambda$ may map to the Hilbert–Pólya operator $H_K$, suggesting conditional equivalences between our $\mathsf{Gen4}$ logic and the Riemann hypothesis:
\begin{itemize}
\item $\mathsf{Gen4}$ slots may map to zeta-function evaluations: $\Phi^{HP}(\operatorname{slot}_0(t)) = \Xi_K(s_0(t))$ and $\Phi^{HP}(\operatorname{slot}_3(t)) = \Xi_K(1-s_0(t))$
\item Logical equalities may become functional equations: $\equiv_B \mapsto \Xi_K(s) = \Xi_K(1-s)$
\item The spectral gap $\gamma(\Lambda)$ may correspond to the gap between consecutive zeta zeros
\item The $\equiv_\star$ equality (reversible computation) may map to unitary equivalence in the operator algebra
\end{itemize}

\textbf{Status:} Under $\Phi^{HP}$ and standard regularity, the transfer operator's spectral gap tracks the $\zeta$-spectrum in the familiar sense; our mechanised checks support the transfer-operator side on the logic ledger.

The transfer operator and symmetric/skew split are given by:
\[
T_\Lambda=\sum_{n,m} Z_{n,m}(\vec q(\Lambda))|n\rangle\langle m|,\qquad
\hat H_{\rm HP}=\tfrac12(T_\Lambda+T_\Lambda^\dagger)+\tfrac{i}{2}(T_\Lambda-T_\Lambda^\dagger).
\]
The spectral gap is defined as $\Delta=\inf_{\lambda\in\sigma(\hat H_{\rm HP})}|\mathrm{Re}\,\lambda|$.

\textbf{Note:} This connection is conjectural and requires additional assumptions about the morphism $\Phi^{HP}$. It does not claim resolution of the Riemann hypothesis. The relevant cross-domain dictionary is collected once in Table~\ref{tab:universal-domain-translation}.
\end{conjecture}

\subsection{Spectral Gap and Information Theory}

\begin{definition}[Spectral Gap as Information Measure via Equality Hierarchy]
\label{def:spectral-gap-info-lbr}
The spectral gap measures information within the L/B/R structure: kernel spectrum (reversible computations respecting $\equiv_\star$ equality), co-kernel spectrum (irreversible computations respecting $\equiv_{\text{meta}} \setminus \equiv_\star$), gap (difference between reversible and irreversible). The information content can be quantified as:
\[
I(\Lambda) = \log \frac{1}{\gamma(\Lambda)} = -\log \gamma(\Lambda)
\]
where larger gaps correspond to more information preservation respecting the equality hierarchy.
\end{definition}

\begin{theorem}[Information-Theoretic Classification via L/B/R Structure]
\label{thm:info-classification-lbr}
Spectral gap classifies systems within the L/B/R structure: converging spectrum (reversible respecting $\equiv_\star$), diverging spectrum (irreversible respecting $\equiv_{\text{meta}} \setminus \equiv_\star$), marginal spectrum (undecidable respecting $\equiv_B \setminus \equiv_\star$).
\end{theorem}

\subsection{Domain Map Generality}

\begin{proposition}[Domain Map Generality via L/B/R Structure (conditional on domain-map axioms)]
\label{prop:domain-generality-spectral-lbr}
The results are general for any domain map within the L/B/R structure - if the logic checks out. Specifically:
\begin{itemize}
\item Any domain map from computation to a mathematical structure preserves the complexity classification respecting the equality hierarchy
\item The Blum axioms hold in any domain where the $\mathsf{Gen4}$ primitive structure is well-defined
\item The spectral gap classification applies universally across domains via $\equiv_\star, \equiv_B, \equiv_{\text{meta}}$
\end{itemize}
\textbf{Note:} This proposition is conditional on the domain-map axioms and should not be over-generalised.
\end{proposition}

\subsection{Mass-Gap Theorem Connection}

\begin{remark}[Mass-Gap Theorem via $\mathsf{Gen4}$]
\label{rem:mass-gap-spectral-gen4}
The spectral gap in the $\mathsf{Gen4}$ primitive spectrum is directly connected to the mass-gap theorem of quantum field theory:
\begin{itemize}
\item Mass gap: The difference between the ground state and first excited state respecting $\equiv_\star$ equality
\item Spectral gap: The difference between kernel and co-kernel eigenvalues respecting $\equiv_{\text{meta}} \setminus \equiv_\star$
\item Both gaps measure the stability of the respective systems within the L/B/R structure
\end{itemize}
The implicit functional arguments $z, \bar{z}$ encode presentation gauges that are factored out in all spectral analyses (see Section~\ref{sec:computation-paradigms}).
\end{remark}

\subsection{Applications to Number Theory and Function Theory}

\begin{conjecture}[Number Theory Applications via L/B/R Structure (Programme)]
\label{conj:number-theory-lbr}
\textbf{Programme:} Under explicit operator construction and spectral analysis assumptions, the spectral gap framework may provide applications to number theory within the L/B/R structure:
\begin{itemize}
\item Riemann hypothesis: May be equivalent to non-trivial spectral gap respecting $\equiv_\star$ equality (requires construction of self-adjoint operator with trace-class resolvent)
\item L-functions: May correspond to different $\mathsf{Gen4}$ primitive spectra (requires explicit spectral correspondence)
\item Modular forms: May arise from RG flow fixed points respecting $\equiv_B$ equality (requires modularity proof)
\end{itemize}

\textbf{Status:} These remain programmeme statements until explicit operator constructions and spectral analyses are provided. The current formulation lacks the necessary operator-theoretic foundations for number theorists to verify.
\end{conjecture}

\begin{theorem}[Function Theory Applications via Equality Hierarchy]
\label{thm:function-theory-lbr}
The spectral gap framework provides applications to function theory via the equality hierarchy:
\begin{itemize}
\item Analytic functions: Correspond to converging RG flow respecting $\equiv_\star$ equality
\item Meromorphic functions: Correspond to marginal RG flow respecting $\equiv_{\text{loc}}$ equality
\item Transcendental functions: Correspond to diverging RG flow respecting $\equiv_{\text{meta}} \setminus \equiv_\star$
\end{itemize}
\end{theorem}

\subsection{Spectral Gap and Computational Complexity}

The connection to computational complexity emerges through the domain morphism $\Psi^{cl}$ from our $\mathsf{Gen4}$ logic to complexity spectral algebras. This morphism provides the foundation for understanding P vs NP through spectral decomposition.

\begin{definition}[Complexity Spectral Morphism $\Psi^{cl}$]
\label{def:complexity-spectral-morphism}
The morphism $\Psi^{cl}$ maps our $\mathsf{Gen4}$ logic to a complexity spectral algebra $\mathcal{C}$:
\begin{itemize}
\item Target algebra: $\mathcal{C} = \mathcal{B}(\mathcal{H})$ with distinguished closed subspace $\mathcal{H}_P$ (representing "P")
\item Orthogonal complement: $\mathcal{H}_{\neg P}$ (representing non-P)
\item Spectral projectors: Separate $\mathcal{H}$ into "P" vs "everything else"
\item Blum axioms: Encoded as operator algebra properties for complexity measures
\end{itemize}
The morphism maps logical invariants to complexity measures, with $\mathsf{Gen4}$ slots becoming projectors onto $\mathcal{H}_P$ or $\mathcal{H}_{\neg P}$. The implicit functional arguments $z, \bar{z}$ encode presentation gauges (see Section~\ref{sec:computation-paradigms}).
\end{definition}

\begin{theorem}[Gap Theorem in Logic via $\Psi^{cl}$]
\label{thm:gap-theorem-logic}
The logic already proves a gap theorem through the $\equiv_\star$ equality that corresponds to complexity separation under $\Psi^{cl}$:
\begin{itemize}
\item Kernel (of $\Psi^{cl}$): Largest subspace mapped to zero, corresponding to polynomial time (P)
\item Cokernel: Quotient space capturing everything else (NP-hard, exponential, etc.)
\item Spectral decomposition: Separates into $\equiv_\star$-invariant vs $\equiv_\star$-non-invariant components
\item The invariant piece is precisely the P-subspace
\end{itemize}
This gap theorem is provable inside the logic using only the $\equiv_\star$ axioms, suggesting a potential logical route to P vs NP classification under additional assumptions about the morphism $\Psi^{cl}$.
\end{theorem}

\begin{theorem}[Spectral Gap and Complexity via L/B/R Structure]
\label{thm:spectral-complexity-lbr}
The spectral gap may determine computational complexity through the $\Psi^{cl}$ morphism:
\begin{itemize}
\item Non-trivial gap: May correspond to efficient computation (P problems) respecting $\equiv_\star$ equality
\item Trivial gap: May correspond to inefficient computation (NP problems) respecting $\equiv_{\text{meta}} \setminus \equiv_\star$
\item No gap: May correspond to undecidable computation respecting $\equiv_B \setminus \equiv_\star$
\end{itemize}
The spectral decomposition under $\Psi^{cl}$ may provide a machine-checkable classification of computational complexity, subject to additional assumptions about the morphism.
\end{theorem}

The spectral gap theorem provides a unifying framework for understanding the deep connections between computation, logic, and physics within the L/B/R structure. It demonstrates how our renormalisation approach can address fundamental questions across multiple domains of mathematics and science through the equality hierarchy.

\begin{abstract}
Fundamental science faces a combinatorics explosion: multiple theories connect to multiple experiments through complex, often convoluted pipelines. The bottleneck lies not in theory or experiment individually, but in their efficient connection. We demonstrate that logic—the greatest common denominator of physics, computation, and mathematics—provides a systematic solution through AI-constructed formal systems with built-in consistency checks.  

We construct a novel interacting positive logic system with six-layer equality hierarchy ($\equiv_L$, $\equiv_B$, $\equiv_R$, $\equiv_{\text{loc}}$, $\equiv_{\text{meta}}$, $\equiv_\star$) and complete BNF specification. The system features boundary semirings ($\oplus_L$, $\otimes_L$; $\oplus_R$, $\otimes_R$), bulk log-semiring ($\oplus_B$, $\otimes_B$), braided duals ($\text{ad}_i$, $F_{ij}$), and the $\mathsf{Gen4}$ primitive with normalization axioms. Truth emerges as a fixed point under renormalization group flow, providing computational semantics through the L/B/R structure. 

There are very close parallels between the construction of formal systems and the construction of QFT theories; both involve a choice of coordinates for one: essentially free fields for QFT and free logic symbols, variables, constants axioms and satisfaction relations for logic. Interactions in logic are traditionally expressed through definitions; in physics through coupling constants. The main difference is that logic is inherently exact by construction - if it is consistent. Consistent truth systems are severely restricted by powerfull, well-known theorems due to Gödel, Löb, Turing and Rice. These express fundamental constraints on maps of logic systems to themselves. These maps can not be total.  

As a first step towards an operational theory of logic, we explore the theory of computation. Computation can be understood as the composition of encoding, function application(s), decoding as well as normalisation. Classically, normalisation can occur at any stage (i.e. commutes as an operator); in the quantum case this is the measurement step that defines the outcome. Note that this is just the basic "compiler" pattern of the theory of programming languages. 

The Turing, Church, and Feynman paradigms emerge as partial representations of the same generating functional through a "four-to-four" renormalization process. We introduce $(3+1)$ natural regulators including an overall scale parameter, extending the RG equation to commuting flows. The resulting logic transformer exhibits self-duality, splitting into input/output dual pieces, with its kernel naturally related to true theorems and spectrum revealing a symmetry-related gap between reversible and irreversible computation. 

We establish consistency through computer algebra implementations in Agda, Coq, Isabelle, and Metamath, generalizing the Curry-Howard-Lambek correspondence to irreversible computation. The logic transformer's hierarchical deformation creates asymmetric equality (implication weighted by local weights), with undeformed truth emerging as the kernel of natural ternary operators combining into arity $(2,2)$ logic transformers—polymorphic generalizations of scaling operators that are intricately self-dual. 


The system exhibits two boundaries-with-direction and partial natural maps constructed using the logic transformer as correlator, naturally interpreted as holographic renormalization in dS/CFT context. We prove a theorem generalizing Rice's theorem, with Gödel, Löb, Turing, and Tarski results as corollaries, plus Noether theorem analogues interpreted as consequences of global invariance of logic presentation and local invariance of subsystem interpretations. 



Applications include LLM training scaling laws and the double dip phenomenon—discovery of effective partial representations through lossy compression. We model large language models as controllable extensions of formal system convolutions, proposing RG flow-aligned stability analysis tools for model-independent language model characterization. Domain morphisms connect to Hilbert-Polya operators (Riemann hypothesis), complexity spectral algebra (P vs NP), and G6 modal convolution (universal LLM invariants).

\end{abstract}

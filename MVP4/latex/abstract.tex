\begin{abstract}
Experimental results are the bedrock offundamental science. In practice there are typically large tapestries of techniques to connect multiple theories to multiple experimental results, leading to combinatorial explosion. In particle physics for instance a wealth of theories, techniques, formalisms and results exist based on various different approaches, paired with a wealth of experimental data. As the current pipelines are long and sometimes convoluted, we propose to use mathematical logic to connect the worlds of theory and of calculation more efficiently and effectively, and show this program is suprisingly powerful.  

Logic is the greatest common denominator between the already closely aligned fields of physics, computation and mathematics. Logic is after all designed to provide consistent models independent of domain application. In this article we show results of using AI driven software engineering to construct logic systems systematically as models of interesting mathematics and physics. 

A "logic" can be defined as a formal system with a notion of truth through a satisfaction relation. Formal systems can be formulated and checked inside mature computer algebra system as programs, and satisfaction relations may be modelled inside the formal system. Even better, formal systems can be formulated as complete formal languages with some constraints on their expressions and one or more inference rules that allows to construct a new sentence from two or more sentences. Since it is known how to construct programming languages for arbitrary digital infrastructure, this yields a natural approach to leverage AI that provides a in-build sanity check on LLM "reasoning" through mathematical logic. 

There are very close parallels between the construction of formal systems and the construction of QFT theories. Both involve a choice of coordinates: essentially free fields for QFT and free alphabet (logic symbols, variables, constants), grammar rules, axioms and satisfaction relations for logic. The main difference is that logic is inherently exact by construction if it is consistent. Different models of a logic provide different views on essentially the same theory. However, a model also defines the logic - which leads to much semantic confusion about "self-reference". 

Consistent truth systems are severely restricted by powerful, well-known theorems due to Gödel, Löb, Turing and Rice. These express fundamental constraints on self-referential maps of logic systems. To get a handle on this we construct a three-sublogic system inside a constructive logic which has no negation and hence no inconsistency (known as "explosion"). Provocatively, we call the subsystems Left, Bulk and Right and refer to the global logic as Meta. We show we can build a logic system based on three tiny "local" semirings with a partial syntactic triality symmetr reminiscent of the octonions. We can construct a partial guarded negation for each of the sublogics by using maps inside the logic itself that are feature in the diagonal lemma. 

Motivated by analytic S-matrix considerations we first explore the theory of computation. Computation can be understood as the composition of encoding, function application(s), decoding as well as normalisation. Classically, normalisation can occur at any stage (i.e. commutes as an operator); in the quantum case this is the measurement step that defines the outcome. Since formal systems are basically formal languages with extra structure, this is the basic "compiler" pattern of the formal theory of programming languages, enriched with a computer algebra point of view for term normalisation. 

We argue that the classic Turing, Church and Feynman views on computation can be modelled as different "partial" domain maps of the same basic logic theory united through a generating functional point of view. The explicit parameter maps from the logic to a specific choice of Turing machine for instance are identified as a form of "renormalisation conditions". With this identification, renormalisation semantics aligns naturally with the local logic subsystems we construct. The novel generating functional approach is deeply related to path integrals and CFT conformal blocks. We argue these are literal analogues by interpreting the AGT correspondence as different stable domain maps from the same logic theory. These are related through computational universality. 

To make the formal generating functional approach well-defined (finite), regulation has to be introduced. We introduce (3+1) natural regulators, including an overall "scale" parameter. We show there is a precise analogue of the usual renormalisation programme, including an extension of the RG equation to several commuting flows. This includes the use of normalisation conditions to fix the parameters to their semantic interpretation as natural numbers (modelled by a semiring). 

To validate the claims above a concrete logic system is constructed. A logic system describing Church-style computation is known to exist - this is the content of the Curry-Howard-Lambek correspondence. We argue we have here a generalisation of this correspondence to irreversible computation or, in other words, the general theory of computing systems. If our construction through computer algebra is consistent, then, we argue, this construction proves the conjecture. Verifying this is non-trivial, as the needed computer algebra is  involved. For safety, we provide full implementations in multiple proof theory languages (agda, coq, isabelle) as well as a partial implementation in metamath. Architecturally we construct 

In logic we argue notions of regularisation and renormalisation arise from a conservative extension of basically any first order logic. This extension can be understood as a natural consistent "twist" of the truth system. Bascially we introduce a modulus variable that models an axiom that could be added to the system - an open choice. This modulus plays the role of a bare parameter that gets fixed to domain semantics using the local logic. In other words, we equate renormalisation conditions with conservative extensions of a bare logic, and argue that the resulting analogue of renormalised couplings is a natural generating functional of invariants. 

The twist changes the truth system to a fundamentally asymmetric notion of equality with "weighted" satisfaction relations. We relate the undeformed truth to the "kernel" of a natural operator inside a domain. The logic also provides the natural co-kernel spectrum. Using this construction both "operator" as well as the "state" views can be realised in any domain map, but not cannot necessarily be related inside the domain. Within the local non-positive domain these are usually not relatable without collapsing the local domain - we prove a kernel separation and classification property that is intimately related to computational complexity measures. At core, this is the difference between kernel and co-kernel of the same operator. We argue this gap is universal. 

Since logic is ubiquitous in science, there are very many applications of the construction inside this paper. The vast majority of these appear to be very very well-known facts within their domain - showcasing the raw power of logic. Some new results are possible though. For instance, we argue that in the domain map to computation, the kernel gap explains the difference between reversible and irreversible computatio, which in the domain map to physics could correspond to an arrow of time. Moreover, the same gap relates even more directly to the theory of computational complexity. 

In a domain map to analytic number theory, we recover a logical variant of the Hilbert-Polya construction and show how generalised-Riemann-Hilbert shaped theorems arise as metatheorems. This opens a direct route to a machine-verifiable proof. We provide extensive code in multiple formal languages. A In the domain map to the theory of computation the same code provides extensive evidence for NP != coNP. As this amounts to a re-axiomatisation of mathematics itself, the verification burden is very high. Conjecturally, in physics the gapped spectrum relates to the existence of a mass gap in quantum field theory. We provide a novel axiomatisation of analytic S-matrix theory that supports any cosmological constant in line with this. 

We show there is a natural notion of two boundaries-with-direction that arise inside the construction for any of the three sublogics we construct. We show that there are two partial natural maps that can be constructed. Holographic renormalisation in the dS/CFT context is a natural interpretation of this, and this motivates the question if we can construct a system that admits two subsystems that are each other's boundaries as natural mirrors. The logic system we construct is exactly this system, at the threshold of logic consistency. In the renormalisation semantics this corresponds to showing the formal system is "renormalisable" - a notion intimately tied to the existence of a system wide truth system. 


Finally we briefly discuss the use of the logic we construct to study more general first order logic systems as convolutions of the basic formal system. This leads to a natural hierarchy of first order logics, and a natural way to study the relationship between different first order logics. In the computation domain map this leads to a theory of software applications that has a very natural application to large language models - the weights of the convolution are mathematically precise models of the weights of the language model. This basically expresses natural language as a weighted convolution of a basic formal language model that, as a combined logic system, can be understood as an effective formal model. We derive some first applications of this for model-independent analysis of LLMs from a stability analysis of the RG flow equations. Some speculation for explanations of the double dip phenomenon, as well as training scaling laws are offered.
\end{abstract}

\begin{abstract}
Experimental results are formally dominant in fundamental science. In practice there are typically large tapestries of techniques to connect multiple theories to multiple experiments, leading to combinatorial explosion. In particle physics for instance a wealth of theories, techniques, formalisms and results exist based on various different approaches, paired with a wealth of experimental data. The actual bottleneck to progress seems to be connecting these two worlds computationally more efficiently and effectively as the current pipelines are long and sometimes convoluted. We propose to use mathematical logic to connect the worlds of theory and of calculation more efficiently and effectively.  

Logic is the greatest common denominator between the already closely aligned fields of physics, computation and mathematics. Logic is after all designed to provide consistent models independent of domain application. In this article we show results of using AI driven software engineering to construct logic systems systematically as models of interesting mathematics and physics. 

A "logic" can be defined as a formal system with a notion of "truth" through a satisfaction relation. Formal systems can be formulated and checked inside mature computer algebra system as programs, and satisfaction relations may be modelled inside the formal system.  Even better, formal systems can be formulated as complete formal languages with some constraints on their expressions and one or more inference rules that allows to construct a new sentence from two or more sentences. Since it is known how to construct programming languages for arbitrary digital infrastructure, this yields a natural approach for AI that provides a in-build sanity check on LLM "reasoning" through mathematical logic. This article shows this is a viable path towards "effective" symbolic reasoning. 

There are very close parallels between the construction of formal systems and the construction of QFT theories. Both involve a choice of coordinates: essentially free fields for QFT and free alphabet (logic symbols, variables, constants), grammar rules, axioms and satisfaction relations for logic. The main difference is that logic is inherently exact by construction if it is consistent. Different models of a logic provide different views on essentially the same theory. However, a model also defines the logic - which leads to much semantic confusion. Consistent truth systems are severely restricted by powerful, well-known theorems due to Gödel, Löb, Turing and Rice. These express fundamental constraints on self-referential maps of logic systems which we interpret here as constraints on domain maps from a constructive logic fragment which cannot have paradoxes by design that contains three related sublogics that have guarded negation. We argue the domain maps from the sublogics to any domain have to be partial. We use AI to provide hypothesis in the constructive logic about the local fragments we prove using computer algebra. This pipeline provides mathematically consistent and coherent models of, basically, any less formal model within science.

Motivated by analytic S-matrix considerations we first explore the theory of computation. Computation can be understood as the composition of encoding, function application(s), decoding as well as normalisation. Classically, normalisation can occur at any stage (i.e. commutes as an operator); in the quantum case this is the measurement step that defines the outcome. Since formal systems are basically formal languages with extra structure, this is the basic "compiler" pattern of the formal theory of programming languages, enriched with a computer algebra point of view for term normalisation. 

We argue that the classic Turing, Church and Feynman views on computation can be modelled as different "partial" domain maps of the same basic logic theory united through a generating functional point of view. The explicit parameter maps from the logic to a specific choice of Turing machine for instance are identified as a form of "renormalisation conditions". With this identification, renormalisation semantics aligns naturally with the local logic subsystems we construct. The novel generating functional approach is deeply related to path integrals and CFT conformal blocks. We argue these are literal analogues by interpreting the AGT correspondence as different stable domain maps from the same logic theory. These are related through computational universality. 


To make the formal generating functional approach well-defined (finite), regulation has to be introduced. We introduce (3+1) natural regulators, including an overall "scale" parameter. We show there is a precise analogue of the usual renormalisation programme, including an extension of the RG equation to several commuting flows. This includes the use of normalisation conditions to fix the parameters to their semantic interpretation as natural numbers (modelled by a semiring). 

To validate the analysis a concrete logic system is constructed. A logic system describing Church-style computation is known to exist - this is the content of the Curry-Howard-Lambek correspondence. We argue we have here a generalisation of this correspondence to irreversible computation or, in other words, the general theory of computing systems. If our construction through computer algebra is consistent, then, we argue, this construction proves the conjecture. Verifying this is non-trivial, as the needed computer algebra is somewhat involved. For safety, we provide full implementations in multiple proof theory languages (agda, coq, isabelle) as well as a partial implementation in metamath.

In logic we argue notions of regularisation and renormalisation arise from a conservative extension of basically any first order logic. This extension can be understood as a natural consistent twist of the truth system. The twist changes the truth system to a fundamentally asymmetric notion of equality with "weighted" satisfaction relations. We relate the undeformed truth to the "kernel" of a natural operator inside a domain. The logic also provides the natural co-kernel spectrum. Using this construction both "operator" as well as the "state" views can be realised in any domain map. Within the local non-positive domain these are usually not relatable without collapsing the local domain - we prove a kernel separation and classification property that is intimately related to computational complexity measures. At core, this is the difference between kernel and co-kernel of the same operator. We argue this gap is universal. 

Since logic is ubiquitous in science, there are very many applications of the construction inside this paper. Actually, the vast majority of these appear to be very very well-known facts within their domain. Some new results are possible though. For instance, we argue that in the domain map to computation, the kernel gap is the difference between reversible and irreversible computation, and show a direct connection to the theory of computational complexity. In the domain map to number theory, this relates directly to the Hilbert-Polya construction. Our mechanised framework yields a spectral/transfer-operator scaffolding that is compatible with Hilbert-Pólya-type operators and with Blum-style internal complexity measures. Multiple mechanised encodings agree on large constructive fragments. Conditioned on the domain morphisms and regularity hypotheses stated in the body, the resulting spectral picture is consistent with GRH-like statements; full proofs remain open. Conjecturally, in physics this relates to the existence of a mass gap in quantum field theory and we make some observations about the relation between positive spectral properties and the arrow of time. In information theory, the spectral gap seems to relate to the basis of fundamental "partial" compression.

We show there is a natural notion of two boundaries-with-direction that arise inside the construction for any of the three sublogics we construct. We show that there are two partial natural maps that can be constructed. Holographic renormalisation in the dS/CFT context is a natural interpretation of this, and this motivates the question if we can construct a system that admits two subsystems that are each other's boundaries as natural mirrors. The logic system we construct is exactly this system, at the threshold of logic consistency. In the renormalisation semantics this corresponds to showing the formal system is "renormalisable" - a notion intimately tied to the existence of a system wide truth system. 

We provide extensive computer algebra internal unit tests. Moreover, as integration tests we derive a swath of cross-checks with well-known results and conjectures in the literature. For instance, a theorem inside the system is proven that generalises the Rice theorem. As corollaries, Gödel, Löb, Turing and Tarski results follow, as well as some well-known results in the CS literature. We argue for an analog of the Noether theorem (both theorems as well as the converse Noether Theorem), which we interpret as the consequence of global invariance of the logic of its presentation, and of local invariance of the logic of one of the sub-systems of its own interpretations. 

Finally we briefly discuss the use of the logic we construct to study more general first order logic systems as convolutions of the basic formal system. This leads to a natural hierarchy of logics, and a natural way to study the relationship between different logics. In the computation domain map this leads to a theory of software applications that has a very natural application to large language models - the weights of the convolution are mathematically precise models of the weights of the language model. This basically expresses natural language as a weighted convolution of a basic formal language model that can be understood as an effective formal model. We derive some first applications of this for model-independent analysis of LLMs from a stability analysis of the RG flow equations. Some speculation for explanations of the double dip phenomenon, as well as training scaling laws are offered.
\end{abstract}

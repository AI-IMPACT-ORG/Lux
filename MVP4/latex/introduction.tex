\section{Introduction}
\label{sec:introduction}

The bottleneck is linking theory$\leftrightarrow$computation efficiently. Our invariant: L/B/R triality with "bulk = two boundaries" + RG-truth as fixed point. See §2 for the universal map.  

 
\subsection{Models of logic}

In mathematics, a "bounty of models of logic" exists because logic results should be independent of presentation. A "model" = structure interpreting the signature; a "presentation" = generators/relations proof calculus. Unlike physics, there is no Lagrangian for logic—the structure itself provides the dynamics.

Consider Boolean algebra as a crisp example. It requires only four axioms to capture all logical operations:
\begin{enumerate}
\item \textbf{Commutativity}: $a \vee b = b \vee a$ and $a \wedge b = b \wedge a$
\item \textbf{Associativity}: $(a \vee b) \vee c = a \vee (b \vee c)$ and $(a \wedge b) \wedge c = a \wedge (b \wedge c)$
\item \textbf{Distributivity}: $a \wedge (b \vee c) = (a \wedge b) \vee (a \wedge c)$ and $a \vee (b \wedge c) = (a \vee b) \wedge (a \vee c)$
\item \textbf{Complementarity}: $a \vee \neg a = 1$ and $a \wedge \neg a = 0$
\end{enumerate}

From these four axioms, one derives the entire structure including De Morgan's laws and all logical equivalences. This demonstrates the profound efficiency of logical systems: minimal foundational assumptions generate maximal mathematical structure. The constructive core provides the normative foundation; see Appendix~\ref{app:mathematical-background} for the complete four-axiom list.


\subsection{Generating functions}

Can we internalise generating functionals as formal power series such that they generate consistent local sublogics while preserving global consistency? The variables $(z,\bar z)$ are presentation gauges and eliminable; all comparisons are modulo qmask (default: phase). See §4 for formal vs analytic usage.

It is known that self-reference leads to paradoxes. Most famously Gödel's incompleteness theorem \cite{godel1931}, Tarski's undefinability \cite{tarski1936}, Löb's theorem \cite{lob1955}, and the diagonal lemma. These constrain domain maps but are used only as constraints later.

We answer by constructing Gen4 (§3) and an RG semantics (§5–§8). 


\subsection{About this paper}

Mechanised proofs: see App. A; hypotheses always stated. 

\begin{notation}[Notation and Standing Assumptions (Ledger)]
\label{not:notation-ledger}
\centering
\footnotesize
\begin{tabular}{|l|l|}
\hline
\textbf{Symbol} & \textbf{Meaning} \\
\hline
$\vec q=(q_1,q_2,q_3)$ & Grading parameters \\
$\Lambda$ & Scale parameter \\
$(z,\bar z)$ & Presentation gauges (eliminable) \\
$\mu_L, \theta_L, \mu_R, \theta_R$ & Core moduli (scale/phase flows) \\
$\mathcal{G}(z,\bar z;\vec q,\Lambda)$ & Generating functional \\
$\mathcal{Z}_{n,m}(\vec q)$ & Correlator coefficients \\
$[L],[R]$ & Boundary projectors \\
$\nu_{L/R}$ & Observers \\
$\oplus_B$ & Bulk monoidal sum \\
$\equiv_\star$ & Observational equality \\
$\equiv_B$ & Bulk equality \\
$\equiv_{\text{meta}}$ & Meta equality \\
qmask & Quotient mask (default: $\{\text{phase}\}$) \\
$\operatorname{ad}_0\ldots\operatorname{ad}_3$ & Regulators (scheme choices) \\
$a_0\ldots a_3$ & True moduli (flow under RG) \\
RC† & Reversibility Constraint \\
\hline
\end{tabular}

\textbf{Invariant:} bulk = two boundaries (normative Core statement). Cross-refs: §7.12 eliminability; §8 fixed points.
\end{notation} 

\paragraph{Contributions.} We summarise the paper's contributions for a HEP-TH audience:
\begin{itemize}
\item \textbf{Core logic}: triality with \emph{bulk = two boundaries}, conservative auxiliaries $(z,\bar z)$, and an equality hierarchy ($\equiv_\star,\equiv_B,\equiv_{\text{meta}}$) anchored to the Core spec.
\item \textbf{RG semantics}: truth as RG fixed point, with reversible/irreversible fragments made precise via entropy and flow conditions.
\item \textbf{Ports/PSDM}: externalisation of irreversible semantics through ports and the PSDM interface; direction of mapping fixed (logical inconsistency $\Rightarrow$ domain divergence).
\item \textbf{Mechanisation}: Racket core and Agda/Coq emitters; tests reference the spec's "truth theorems".
\item \textbf{Cross-domain map}: a single domain ledger (Table~\ref{tab:universal-domain-translation}) linking computation, physics, learning, and number theory.
\end{itemize}

\paragraph{Normative stance.} We treat the CLEAN v10 Core as normative for triality and bulk = two boundaries, and CLEAN v10 CLASS as normative for the four-moduli parametric normaliser, PSDM, and ports; the present paper develops derived, domain-mapped consequences and supplies compact proofs under explicit, local hypotheses (positivity/contractivity/$\omega$-cpo). Speculative identifications (AGT, Hilbert–Pólya, spectral complexity) are boxed and unused in proofs.

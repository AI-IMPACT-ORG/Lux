\section{Conclusions and Future Work}
\label{sec:conclusion}

We live in an era of information integration. Modern AI tools allow us to access, manipulate, cross-check, and compare information from different sources using natural language informational interfaces. Moreover, AI tooling in software engineering allow us to build even better tools to do all of that. In science, whose core use case is information creation, we should be at the cusp of a new era of scientific discovery. What has been lagging behind is an efficient and effective method to close the loop between information creation and information integration. This paper proposes a method to do just that in the context of formal science, building on already widely available tools.

This paper contains remarkably few truly new ideas, as almost all concepts have been discussed in various forms in quite some detail in the scientific literature. The main new observation in this article is that different areas of science are much more closely related than is usually appreciated when seen through the lens of logic. In order to see this we make use of the statistical evidence gathered in the training of chatbots. This observation can certainly be systematized by building tools that support human insights, using logic scaffolding to ground human reasoning in purpose-built formal systems. For instance, using statistical methods, one can in fact derive "truth" from a suitable set of prompts to a sample of synthetic chatbots, as long as that truth context correlates with the chatbot contexts (plural) in a known way. The results of this article make this even more feasible - the engine is basic standard statistical inference. 

Discussions of logic tend to involve discussions of philosophy at some point. This paper adopts the common instrumentalist view pervasive in modern physics. That is not to say that the philosophy angle is not interesting. For instance, the system of logic constructed here supports a view that logic systems are fundamentally "open" systems, and need a boundary to even be defined. When translated into an observer-style physics this leads to familiar discussions about the interpretation of quantum mechanics, but now with a twist: if any system fundamentally needs a boundary, what is the boundary of the universe? Or in a different interpretation: who observes the universe when there is no internal observers to observe it? These are at some level not questions of logic but of the interpretation of logic. I.e. of semantics. The author interprets the systematic applications of logic in this paper as evidence of fundamental mathematical structure.




% Computational Complexity Theory
% Missing elements from the abstract - Blum axioms, P vs NP, Hilbert-Polya

\section{Computational Complexity Theory}

\subsection{Blum Axioms for Generating Functionals}

\begin{definition}[Blum Axioms]
\label{def:blum-axioms}
The Blum axioms for computational complexity are:
\begin{enumerate}
\item \textbf{Blum Axiom 1}: For any program $P$ and input $x$, the complexity measure $\Phi_P(x)$ is defined if and only if $P(x)$ halts
\item \textbf{Blum Axiom 2}: The complexity measure $\Phi_P(x)$ is a total computable function
\item \textbf{Blum Axiom 3}: For any computable function $f$, there exists a program $P$ such that $\Phi_P(x) = f(x)$ for all $x$
\end{enumerate}
\end{definition}

\begin{theorem}[Blum Axioms for Generating Functionals]
\label{thm:blum-generating}
The Blum axioms hold for the generating functional of invariants. Specifically:
\begin{enumerate}
\item For any computational process encoded in $G(z, \bar{z}; \vec{q}, \Lambda)$, the complexity measure $\Phi_G$ is defined if and only if the RG flow converges
\item The complexity measure $\Phi_G$ is computable through the generating function structure
\item For any computable complexity function $f$, there exists a generating function $G$ such that $\Phi_G = f$
\end{enumerate}
\end{theorem}

\subsection{P vs NP Classification}

\begin{theorem}[P vs NP via Logic Transformer Spectrum]
\label{thm:p-vs-np-spectrum}
P vs NP reduces to a topological classification of the logic transformer spectrum. Specifically:
\begin{itemize}
\item \textbf{P problems}: Correspond to eigenvalues in the kernel of the logic transformer (reversible computations)
\item \textbf{NP problems}: Correspond to eigenvalues in the co-kernel of the logic transformer (irreversible computations)
\item \textbf{P = NP}: Equivalent to the spectral gap between kernel and co-kernel being trivial
\end{itemize}
\end{theorem}

\subsection{Spectral Gap and Computational Complexity}

\begin{definition}[Spectral Gap]
\label{def:spectral-gap}
The spectral gap of the logic transformer is the difference between:
\begin{itemize}
\item \textbf{Kernel spectrum}: Eigenvalues corresponding to reversible computations
\item \textbf{Co-kernel spectrum}: Eigenvalues corresponding to irreversible computations
\end{itemize}

The gap measures the computational complexity of the system.
\end{definition}

\subsection{Hilbert-Polya Operator}

\begin{definition}[Hilbert-Polya Operator]
\label{def:hilbert-polya}
In the domain map to theory of functions on number fields, the logic transformer provides a natural Hilbert-Polya operator $\mathcal{H}$ such that:
\[
\mathcal{H} \phi = \lambda \phi
\]
where the eigenvalues $\lambda$ correspond to the zeros of the Riemann zeta function.
\end{definition}

\subsection{Zeta-Function Interpretation}

\begin{theorem}[Zeta-Function Interpretation]
\label{thm:zeta-interpretation}
The natural heat-kernel regulator provides a zeta-function interpretation:
\[
\zeta_{\mathcal{H}}(s) = \text{Tr}(\mathcal{H}^{-s}) = \sum_{\lambda} \lambda^{-s}
\]
where:
\begin{itemize}
\item The zeros of $\zeta_{\mathcal{H}}(s)$ correspond to eigenvalues of $\mathcal{H}$
\item The Riemann hypothesis is equivalent to the spectral gap being non-trivial
\item The logic transformer spectrum determines the distribution of zeros
\end{itemize}
\end{theorem}

\subsection{Hilbert-Polya Scenario}

\begin{theorem}[Hilbert-Polya Scenario]
\label{thm:hilbert-polya-scenario}
The construction conforms to the Hilbert-Polya scenario:
\begin{itemize}
\item The logic transformer provides the Hilbert-Polya operator
\item The eigenvalues correspond to zeros of the zeta function
\item The spectral gap determines the distribution of zeros
\item The Riemann hypothesis follows from the non-triviality of the spectral gap
\end{itemize}
\end{theorem}

\subsection{Mass-Gap Theorem Connection}

\begin{remark}[Mass-Gap Theorem]
\label{rem:mass-gap}
The spectral gap in the logic transformer spectrum is directly connected to the mass-gap theorem of quantum field theory:
\begin{itemize}
\item \textbf{Mass gap}: The difference between the ground state and first excited state
\item \textbf{Spectral gap}: The difference between kernel and co-kernel eigenvalues
\item Both gaps measure the stability of the respective systems
\end{itemize}
\end{remark}

\subsection{Domain Maps and Generality}

\begin{theorem}[Domain Map Generality]
\label{thm:domain-generality}
The results are general for any domain map - if the logic checks out. Specifically:
\begin{itemize}
\item Any domain map from computation to a mathematical structure preserves the complexity classification
\item The Blum axioms hold in any domain where the generating function structure is well-defined
\item The spectral gap classification applies universally across domains
\end{itemize}
\end{theorem}

\subsection{Computational Complexity and Information Theory}

\begin{definition}[Complexity-Information Correspondence]
\label{def:complexity-information}
The computational complexity is directly related to information theory through:
\begin{itemize}
\item \textbf{Reversible computations}: Zero information loss, polynomial complexity
\item \textbf{Irreversible computations}: Information loss, exponential complexity
\item \textbf{Undecidable computations}: Infinite information loss, undecidable complexity
\end{itemize}
\end{definition}

\subsection{Complexity Classes and RG Flow}

\begin{theorem}[Complexity Classes via RG Flow]
\label{thm:complexity-rg}
The computational complexity classes correspond to different RG flow behaviors:
\begin{align}
\text{P} &\leftrightarrow \text{Converging RG flow} \\
\text{NP} &\leftrightarrow \text{Diverging RG flow} \\
\text{Undecidable} &\leftrightarrow \text{Marginal RG flow}
\end{align}
\end{theorem}

This framework provides a unified understanding of computational complexity through the generating function structure, connecting it to information theory, quantum field theory, and number theory through the logic transformer spectrum.

% CFT Conformal Blocks and AGT Correspondence
% Missing elements from the abstract - CFT connection and AGT

\section{CFT Conformal Blocks and AGT Correspondence}

\subsection{CFT Conformal Blocks Connection}

\begin{definition}[CFT Conformal Blocks]
\label{def:conformal-blocks}
Conformal blocks are the building blocks of correlation functions in conformal field theory. For a 4-point correlation function:
\[
\langle \phi_1(z_1) \phi_2(z_2) \phi_3(z_3) \phi_4(z_4) \rangle = \sum_p C_{12p} C_{34p} \mathcal{F}_p(z_i)
\]
where $\mathcal{F}_p(z_i)$ are the conformal blocks and $C_{ijk}$ are the structure constants.
\end{definition}

\subsection{Generating Function as Conformal Block}

\begin{theorem}[Generating Function as Conformal Block]
\label{thm:generating-conformal}
The computational generating function $G(z, \bar{z}; q_1, q_2, q_3, \Lambda)$ can be identified with a conformal block in CFT. Specifically:
\[
G(z, \bar{z}; \vec{q}, \Lambda) = \mathcal{F}_{\vec{q}}(z, \bar{z}; \Lambda)
\]
where $\mathcal{F}_{\vec{q}}$ is a conformal block with:
\begin{itemize}
\item External weights determined by the computational registers $z, \bar{z}$
\item Internal weights determined by the grading parameters $\vec{q} = (q_1, q_2, q_3)$
\item Modular parameter $\Lambda$ controlling the scale
\end{itemize}
\end{theorem}

\subsection{Virasoro Algebra and Conformal Blocks}

\begin{definition}[Virasoro Conformal Blocks]
\label{def:virasoro-blocks}
The Virasoro conformal blocks are constructed using the Virasoro algebra generators $L_n$:
\[
\mathcal{F}_h(z) = \langle h | \phi_1(z_1) \phi_2(z_2) | h \rangle
\]
where $|h\rangle$ is a primary state with conformal weight $h$, and the block is computed using the Virasoro algebra:
\[
[L_m, L_n] = (m-n)L_{m+n} + \frac{c}{12}(m^3-m)\delta_{m,-n}
\]
\end{definition}

\subsection{AGT Correspondence}

\begin{definition}[AGT Correspondence]
\label{def:agt-correspondence}
The Alday-Gaiotto-Tachikawa (AGT) correspondence relates:
\begin{itemize}
\item \textbf{4D $\mathcal{N}=2$ gauge theories} on $\mathbb{R}^4$
\item \textbf{2D conformal field theories} on Riemann surfaces
\item \textbf{Conformal blocks} in the CFT
\item \textbf{Instanton partition functions} in the gauge theory
\end{itemize}
\end{definition}

\subsection{AGT and Computational Paradigms}

\begin{theorem}[AGT-Computational Correspondence]
\label{thm:agt-computational}
The AGT correspondence naturally appears in the computational framework through the generating function structure. The three computational paradigms correspond to different limits of the AGT correspondence:

\begin{align}
\text{Turing Machines} &\leftrightarrow \text{Classical limit of AGT} \\
\text{Lambda Calculus} &\leftrightarrow \text{Quantum limit of AGT} \\
\text{Path Integrals} &\leftrightarrow \text{Full AGT correspondence}
\end{align}
\end{theorem}

\subsection{Parameter Mappings in AGT}

\begin{definition}[AGT Parameter Mappings]
\label{def:agt-parameters}
The AGT correspondence provides explicit parameter mappings:

\begin{align}
\text{Gauge coupling } g^2 &\leftrightarrow \text{Scale parameter } \Lambda \\
\text{Mass parameters } m_i &\leftrightarrow \text{Grading parameters } q_i \\
\text{$\Omega$-background } \epsilon_1, \epsilon_2 &\leftrightarrow \text{Computational registers } z, \bar{z} \\
\text{Instanton number } k &\leftrightarrow \text{Virasoro levels } n, m
\end{align}
\end{definition}

\subsection{Conformal Blocks and Computational Weights}

\begin{theorem}[Conformal Blocks as Computational Weights]
\label{thm:blocks-weights}
The computational weights $\mathcal{Z}_{n,m}(\vec{q})$ are proportional to conformal blocks:
\[
\mathcal{Z}_{n,m}(\vec{q}) = \mathcal{F}_{n,m}(\vec{q}) \cdot \Lambda^{-(n+m)}
\]
where $\mathcal{F}_{n,m}(\vec{q})$ are the conformal blocks for the corresponding CFT.
\end{theorem}

\subsection{Toda Hierarchy Connection}

\begin{definition}[Toda Hierarchy]
\label{def:toda-hierarchy}
The Toda hierarchy is an infinite system of integrable differential equations. In the context of AGT, it provides:
\begin{itemize}
\item \textbf{Commuting flows} that correspond to RG flows
\item \textbf{Hamiltonian structure} for the renormalization group
\item \textbf{Integrability} of the RG equations
\end{itemize}
\end{definition}

\subsection{Extended RG Equations}

\begin{definition}[Extended RG Equations]
\label{def:extended-rg}
The extension of RG equations to several commuting flows natural in the Toda hierarchy takes the form:
\begin{align}
\frac{\partial G}{\partial t_1} &= \beta_1(G, \vec{q}, \Lambda) \\
\frac{\partial G}{\partial t_2} &= \beta_2(G, \vec{q}, \Lambda) \\
\frac{\partial G}{\partial t_3} &= \beta_3(G, \vec{q}, \Lambda) \\
\frac{\partial G}{\partial \Lambda} &= \beta_\Lambda(G, \vec{q}, \Lambda)
\end{align}
where $t_i$ are the Toda hierarchy times and the flows commute:
\[
[\frac{\partial}{\partial t_i}, \frac{\partial}{\partial t_j}] = 0
\]
\end{definition}

\subsection{Beta and Gamma Functions}

\begin{definition}[Beta and Gamma Functions]
\label{def:beta-gamma}
The renormalization group equations involve:
\begin{itemize}
\item \textbf{Beta functions} $\beta_i$: 3 functions corresponding to the grading parameters $(q_1, q_2, q_3)$
\item \textbf{Gamma functions} $\gamma$: 1 function corresponding to the overall scale $\Lambda$
\end{itemize}

This creates a fundamental imbalance: 3 beta functions vs 1 gamma function, reflecting the asymmetry in the computational structure.
\end{definition}

\subsection{a-functions and c-functions}

\begin{conjecture}[Generalized a-functions and c-functions]
\label{conj:a-c-functions}
Through the AGT correspondence, we can define natural generalizations of:
\begin{itemize}
\item \textbf{a-functions}: Related to the anomaly coefficients in 4D gauge theory
\item \textbf{c-functions}: Related to the central charge in 2D CFT
\end{itemize}

These are constructed using the natural analog of the Fisher information metric (c-theorem) and provide measures of information flow in the computational system.
\end{conjecture}

\subsection{Conformal Blocks and Information Theory}

\begin{theorem}[Conformal Blocks as Information Measures]
\label{thm:blocks-information}
The conformal blocks $\mathcal{F}_{n,m}(\vec{q})$ serve as information measures in the computational framework:
\begin{itemize}
\item \textbf{Converging blocks}: Correspond to reversible computations (information preserving)
\item \textbf{Diverging blocks}: Correspond to irreversible computations (information destroying)
\item \textbf{Marginal blocks}: Correspond to undecidable computations
\end{itemize}
\end{theorem}

This connection between CFT conformal blocks and computational paradigms provides the mathematical foundation for understanding how the generating function approach unifies computation, logic, and physics through the AGT correspondence.

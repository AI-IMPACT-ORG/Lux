% Complete Renormalization Story
% Comprehensive treatment of renormalization in the computational framework

\section{The Complete Renormalisation Story}
\label{sec:complete-renormalization}

This section provides a comprehensive treatment of renormalisation in the computational framework, integrating all the machinery developed in Sections \ref{sec:S01_computation_invitation} through \ref{sec:computation-revisited}. It serves as the definitive reference for the unified renormalisation isomorphism.

\subsection{The Renormalization Problem}

\begin{definition}[The Renormalization Problem]
\label{def:renorm-problem}
The renormalization problem in the computational framework is:
\begin{itemize}
\item \textbf{Unrenormalised generating functions} may contain divergences or exhibit pathological behavior
\item \textbf{Raw computational weights} $\mathcal{Z}_{n,m}(\vec{q})$ (matrix elements) may not be well-defined
\item \textbf{RG flow} may diverge, oscillate, or fail to converge
\item \textbf{Information} may be lost or destroyed during computation
\end{itemize}
\end{definition}

\subsection{The Renormalization Solution}

\begin{definition}[The Renormalization Solution]
\label{def:renorm-solution}
The renormalization solution provides:
\begin{itemize}
\item \textbf{Regularisation}: Introduction of regulators to make unrenormalised functions well-defined
\item \textbf{RG Flow}: Evolution under $\mathcal{R}_b$ to remove divergences
\item \textbf{Renormalisation Conditions}: Constraints ensuring convergence to fixed points
\item \textbf{Finite Results}: Well-defined, information-preserving computations
\end{itemize}
\end{definition}

\subsection{The Complete Renormalization Procedure}

\begin{definition}[Complete Renormalization Procedure]
\label{def:complete-renorm}
The complete renormalization procedure consists of four steps:

\begin{enumerate}
\item \textbf{Regularisation}: Introduce regulators
   \begin{itemize}
   \item Conceptual regulator: boundary at $\infty$
   \item Operational regulator: scale parameter $\Lambda$
   \item Grading parameters: $(q_1, q_2, q_3)$
   \item State regulator: Virasoro levels $n, m$
   \end{itemize}

\item \textbf{RG Flow Evolution}: Apply RG map $\mathcal{R}_b$
   \begin{align}
   \frac{dG_t}{dt} &= \beta_G(G_t, \vec{q}_t, \Lambda_t) \\
   \frac{d\vec{q}_t}{dt} &= \vec{\beta}_q(G_t, \vec{q}_t, \Lambda_t) \\
   \frac{d\Lambda_t}{dt} &= \beta_\Lambda(G_t, \vec{q}_t, \Lambda_t)
   \end{align}

\item \textbf{Renormalisation Conditions}: Impose convergence criteria
   \begin{itemize}
   \item Converging flow: $\lim_{\Lambda \to \infty} G$ exists and is finite
   \item Information preservation: $\text{ker}(\mathcal{T}) \neq \emptyset$
   \item Unitarity: $\mathcal{R}_b$ preserves inner products
   \end{itemize}

\item \textbf{Removal of Regulators}: Take limits to obtain finite results
   \begin{align}
   G_{\text{ren}} &= \lim_{b \to \infty} (\mathcal{R}_b G_{\text{unren}}) \\
   \mathcal{Z}_{\text{ren}} &= \lim_{b \to \infty} (\mathcal{R}_b \mathcal{Z}_{\text{unren}})
   \end{align}
\end{enumerate}
\end{definition}

\subsection{Renormalization in Different Paradigms}

\subsubsection{Turing Machine Renormalization}

\begin{theorem}[Turing Machine Renormalization]
\label{thm:turing-renorm}
For Turing machines $(q_1=1, q_2=0, q_3=0)$:
\begin{itemize}
\item \textbf{Unrenormalised}: Raw state transition rules that may lead to infinite loops
\item \textbf{Regularisation}: Introduction of halting criterion and state space boundaries
\item \textbf{RG Flow}: Evolution to deterministic, halting computations
\item \textbf{Renormalised}: Well-defined halting computations that preserve information
\end{itemize}
\end{theorem}

\subsubsection{Lambda Calculus Renormalization}

\begin{theorem}[Lambda Calculus Renormalization]
\label{thm:lambda-renorm}
For lambda calculus $(q_1=0, q_2=1, q_3=0)$:
\begin{itemize}
\item \textbf{Unrenormalised}: Raw $\beta$-reduction rules that may not terminate
\item \textbf{Regularisation}: Introduction of normal form conditions and termination criteria
\item \textbf{RG Flow}: Evolution to normal forms that preserve computational meaning
\item \textbf{Renormalised}: Well-defined normal forms that preserve information
\end{itemize}
\end{theorem}

\subsubsection{Path Integral Renormalization}

\begin{theorem}[Path Integral Renormalization]
\label{thm:path-renorm}
For path integrals $(q_1=0, q_2=0, q_3=1)$:
\begin{itemize}
\item \textbf{Unrenormalised}: Raw quantum amplitudes that may violate unitarity
\item \textbf{Regularisation}: Introduction of amplitude normalization and unitarity constraints
\item \textbf{RG Flow}: Evolution to unitary quantum evolution
\item \textbf{Renormalised}: Unitary quantum evolution that preserves probability amplitudes
\end{itemize}
\end{theorem}

\subsection{Information-Theoretic Renormalization}

\begin{definition}[Information-Theoretic Renormalization]
\label{def:info-renorm}
The renormalization procedure preserves information through:
\begin{itemize}
\item \textbf{Converging RG Flow}: Information is preserved, computation is reversible
\item \textbf{Diverging RG Flow}: Information is destroyed, computation is irreversible
\item \textbf{Marginal RG Flow}: Information behavior is undecidable
\end{itemize}
\end{definition}

\subsection{Renormalization and Truth}

\begin{theorem}[Renormalization and Truth]
\label{thm:renorm-truth}
A computational statement is true if and only if it corresponds to converging RG flow:
\begin{align}
\text{True} &\Leftrightarrow \lim_{\Lambda \to \infty} G \text{ converges} \\
\text{False} &\Leftrightarrow \lim_{\Lambda \to \infty} G \text{ diverges} \\
\text{Undecidable} &\Leftrightarrow \lim_{\Lambda \to \infty} G \text{ oscillates}
\end{align}
\end{theorem}

\subsection{Renormalization and Complexity}

\begin{theorem}[Renormalization and Complexity]
\label{thm:renorm-complexity}
The computational complexity is determined by the RG flow behavior:
\begin{align}
\text{Polynomial time} &\Leftrightarrow \text{Converging RG flow} \\
\text{Exponential time} &\Leftrightarrow \text{Diverging RG flow} \\
\text{Undecidable} &\Leftrightarrow \text{Marginal RG flow}
\end{align}
\end{theorem}

\subsection{Renormalization and Universality}

\begin{theorem}[Renormalization and Universality]
\label{thm:renorm-universality}
All three computational paradigms are universal because they all flow to the same fixed point:
\[
\lim_{\Lambda \to \infty} G_{\text{Turing}} = \lim_{\Lambda \to \infty} G_{\text{Church}} = \lim_{\Lambda \to \infty} G_{\text{Feynman}} = G_*
\]
where $G_*$ is the universal fixed point of the RG flow.
\end{theorem}

\subsection{Renormalization and AGT Correspondence}

\begin{theorem}[Renormalization and AGT]
\label{thm:renorm-agt}
The renormalization procedure naturally appears in the AGT correspondence:
\begin{itemize}
\item \textbf{4D Gauge Theory}: Renormalization of instanton partition functions
\item \textbf{2D CFT}: Renormalization of conformal blocks
\item \textbf{AGT Correspondence}: Maps renormalization between the two theories
\end{itemize}
\end{theorem}

\subsection{Renormalization and Logic Transformer}

\begin{theorem}[Renormalization and Logic Transformer]
\label{thm:renorm-logic}
The renormalization procedure is implemented by the logic transformer:
\begin{itemize}
\item \textbf{Kernel}: Reversible computations (converging RG flow)
\item \textbf{Co-kernel}: Irreversible computations (diverging RG flow)
\item \textbf{Spectral Gap}: Measures the efficiency of renormalization
\end{itemize}
\end{theorem}

\subsection{The Complete Picture}

The complete renormalization story reveals that:

\begin{itemize}
\item \textbf{Computation}: All paradigms are unified through the generating function
\item \textbf{Logic}: The logic transformer implements the renormalization procedure
\item \textbf{Physics}: CFT and AGT provide the mathematical framework
\item \textbf{Information}: Preserved or destroyed through RG flow behavior
\item \textbf{Complexity}: Classified through spectral analysis
\item \textbf{Truth}: Defined through converging RG flow
\item \textbf{Universality}: Consequence of RG flow convergence
\end{itemize}

This unified perspective provides a complete understanding of renormalization in the computational framework, revealing the deep structural connections between computation, logic, and physics.

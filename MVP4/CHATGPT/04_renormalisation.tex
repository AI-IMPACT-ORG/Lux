% Definition of Renormalisation
% Extracted from ChatGPT conversation on renormalization logic

\section{Renormalisation: From Raw to Refined}
\label{sec:renormalisation}

The generating function framework introduced in Section \ref{sec:S01_computation_invitation} produces formal expansions that may contain divergences. This section develops the renormalisation isomorphism that transforms raw computational weights $\mathcal{Z}_{n,m}(\vec{q})$ into finite, well-defined quantities.

\subsection{Unrenormalised vs Renormalised Green's Functions}

In the context of the computational generating function framework, we distinguish between unrenormalised and renormalised Green's functions through their relationship to the RG flow developed in Section \ref{sec:rg-flow-behavior} and computational semantics.

\subsection{Unrenormalised Green's Functions}

\begin{definition}[Unrenormalised Green's Functions]
\label{def:unrenormalised-greens}
The unrenormalised Green's functions are the matrix elements $\mathcal{Z}_{n,m}(\vec{q})$ appearing directly in the generating function defined in Section \ref{sec:S01_computation_invitation}:
\[
G_{\text{unren}}(z, \bar{z}; \vec{q}, \Lambda) = \sum_{n,m=0}^{\infty} \frac{z^n \bar{z}^m}{n! m!} \cdot \mathcal{Z}_{n,m}(\vec{q}) \cdot \Lambda^{-(n+m)}
\]
where $\mathcal{Z}_{n,m}(\vec{q})$ are the bare matrix elements without any RG flow corrections.
\end{definition}

The unrenormalised Green's functions represent the raw computational weights before any renormalisation procedure is applied. They may contain divergences or exhibit pathological behavior under RG flow.

\subsection{Renormalised Green's Functions}

\begin{definition}[Renormalised Green's Functions]
\label{def:renormalised-greens}
The renormalised Green's functions are obtained by applying the RG map $\mathcal{R}_b$ to the unrenormalised functions:
\[
G_{\text{ren}}(z, \bar{z}; \vec{q}, \Lambda) = \lim_{b \to \infty} (\mathcal{R}_b G_{\text{unren}})(z, \bar{z}; \vec{q}, \Lambda)
\]
where the RG map acts on the weights as:
\[
(\mathcal{R}_b\mathcal{Z})_{n,m} = b^{-\Delta(n,m)}\,\mathcal{Z}_{\lfloor n/b\rfloor,\lfloor m/b\rfloor}
\]
\end{definition}

The renormalised Green's functions represent the finite, well-defined computational weights after removing divergences through the renormalisation procedure.

\subsection{Renormalisation Procedure}

\begin{definition}[Renormalisation Procedure]
\label{def:renormalisation-procedure}
The renormalisation procedure consists of:
\begin{enumerate}
\item \textbf{Regularisation}: Introduce regulators (boundaries, scale $\Lambda$, grading parameters $\vec q$) to make the unrenormalised Green's functions well-defined
\item \textbf{RG Flow}: Apply the RG map $\mathcal{R}_b$ to evolve the system from scale $\Lambda$ to scale $b\Lambda$
\item \textbf{Renormalisation Conditions}: Impose conditions ensuring convergence to a fixed point
\item \textbf{Removal of Regulators}: Take the limit $b \to \infty$ to obtain finite renormalised Green's functions
\end{enumerate}
\end{definition}

\subsection{Renormalised vs Unrenormalised: Key Differences}

\begin{theorem}[Renormalised vs Unrenormalised Green's Functions]
\label{thm:renormalised-difference}
The key differences between renormalised and unrenormalised Green's functions are:

\begin{enumerate}
\item \textbf{Finite vs Divergent}: Renormalised Green's functions are finite, while unrenormalised ones may diverge
\item \textbf{RG Flow Behavior}: Renormalised functions exhibit converging RG flow, while unrenormalised ones may diverge or oscillate
\item \textbf{Computational Semantics}: Renormalised functions correspond to reversible computations, while unrenormalised ones may correspond to irreversible or undecidable computations
\item \textbf{Information Preservation}: Renormalised functions preserve information, while unrenormalised ones may destroy information
\end{enumerate}
\end{theorem}

\subsection{Renormalisation Across Paradigms}

The renormalisation procedure applies uniformly across all computational paradigms. Traditional computational models represent the renormalised versions of their raw counterparts.

\subsection{Renormalisation Group Fixed Points}

\begin{definition}[RG Fixed Points]
\label{def:rg-fixed-points}
A renormalised Green's function $G_{\text{ren}}$ is at an RG fixed point if:
\[
\mathcal{R}_b G_{\text{ren}} = G_{\text{ren}}
\]
for all $b > 1$. Fixed points correspond to scale-invariant computational processes.
\end{definition}

\subsection{Renormalisation and Truth}

The renormalisation procedure serves as the bridge between the formal mathematical structure and computational truth. The comprehensive treatment is developed in Section \ref{sec:complete-renormalization}.

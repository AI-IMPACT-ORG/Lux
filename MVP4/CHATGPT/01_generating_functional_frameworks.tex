% Definition of Generating Functional Frameworks
% Extracted from ChatGPT conversation on renormalization logic

\section{An Invitation to Computation: Three Views Unified}
\label{sec:S01_computation_invitation}

This section introduces the generating functional framework that unifies three classical paradigms of computation: Turing machines, Church's lambda calculus, and Feynman path integrals. The computational aspects of this unification are developed in Section \ref{sec:computation-revisited}.

We organize computation around a four-step pipeline: encoding → operator application → normalization (regularisation) → decoding. The various computational paradigms emerge as different parameter choices in a generating function $G(z, \bar{z}; \vec{q}, \Lambda)$ that connects computation, logic, and physics through the RG flow isomorphism.

\subsection{The Sextary Connective and Denotation}

\begin{definition}[Sextary connective and denotation]
\label{def:sextary-connective}
Let $\mathcal{L}$ be a logical language with atomic terms denoting elements of a graded Fock space $\mathcal{H}$ and with a sextary connective
\[
\mathsf{G}_6(\cdot,\cdot;\cdot,\cdot,\cdot;\cdot).
\]
For a valuation $\nu$ sending formulas to $\mathcal{H}$ and parameters $(q_1,q_2,q_3)\in\mathbb{C}^3$, $\Lambda\in\mathbb{R}_+$, define
\[
\llbracket \mathsf{G}_6(\phi,\bar\phi;q_1,q_2,q_3;\Lambda)\rrbracket_\nu
\;:=\;
G\big(\llbracket\phi\rrbracket_\nu,\llbracket\bar\phi\rrbracket_\nu;q_1,q_2,q_3,\Lambda\big),
\]
with $G$ as in \eqref{eq:generating-function}.
\end{definition}

The sextary connective $\mathsf{G}_6$ serves as the logical primitive whose denotation is the generating function. This provides the bridge between the logical syntax and the computational semantics.

\subsection{Computational Generating Function}

\begin{definition}[Computational Generating Function]
\label{def:generating-function}
The generating function we consider takes the form:
\begin{align}
\label{eq:generating-function}
G(z, \bar{z}; q_1, q_2, q_3, \Lambda) = \sum_{n,m=0}^{\infty} \frac{z^n \bar{z}^m}{n! m!} \cdot \mathcal{Z}_{n,m}(q_1, q_2, q_3) \cdot \Lambda^{-(n+m)}
\end{align}
where:
\begin{itemize}
\item $z, \bar{z} \in \mathbb{C}$: computational registers storing input/output states
\item $\Lambda \in \mathbb{R}^+$: overall scale parameter (corresponds to main framework's overall coupling strength)
\item $(q_1, q_2, q_3) \in \mathbb{C}^3$: three grading parameters determining computational paradigm (corresponds to main framework's 3 grading structure)
\item $n, m \in \mathbb{N}$: Virasoro levels indexing computational states
\item $\mathcal{Z}_{n,m}(\vec{q})$: computational weights (matrix elements) that encode the dynamics for Virasoro levels $(n,m)$ with three grading parameters $\vec{q}=(q_1, q_2, q_3)$
\end{itemize}
\end{definition}

\subsection{Mathematical Foundation}

The generating function is built on a mathematical foundation involving:

\begin{definition}[Ambient CFT/VOA Structure]
\label{def:ambient-structure}
Fix a unitary representation of $\Vir\oplus\overline{\Vir}$ on a graded Fock space $\mathcal{H}$ with vacuum $|0\rangle$ and central charge $c$. The Virasoro generators satisfy:
\[
[L_m,L_n]=(m-n)L_{m+n}+\frac{c}{12}(m^3-m)\delta_{m,-n}, \quad [L_m,\bar L_n]=0
\]
with $\bar L_n$ satisfying identical relations. The computational weights are matrix elements $\mathcal{Z}_{n,m}(\vec{q}) = \langle n,m|\hat{W}(\vec{q})|0\rangle$ of operators $\hat{W}(\vec{q})$ in this ambient structure. These weights naturally encode CFT conformal blocks through the Virasoro structure.
\end{definition}

The computational Hamiltonian $\hat H_{\text{comp}}$ governs evolution through Virasoro operations. The technical details are developed in Section \ref{sec:complete-renormalization}.


\subsection{Four-Step Computational Pipeline}

The computational process naturally follows a four-step pipeline:
\begin{enumerate}
\item \textbf{Encoding}: Input $(a,b) \mapsto z^a \bar{z}^b$
\item \textbf{Operator Application}: Repeated Virasoro operations under the computational Hamiltonian
\item \textbf{Normalization (Regularisation)}: Regulator choice - detailed in Section \ref{sec:regulator-view}
\item \textbf{Decoding}: Output result extraction
\end{enumerate}

\subsection{Three Computational Paradigms}

The three computational views arise from different values of the grading parameters $\vec{q} = (q_1, q_2, q_3)$: Turing machines $\vec{q} = (1, 0, 0)$, lambda calculus $\vec{q} = (0, 1, 0)$, and path integrals $\vec{q} = (0, 0, 1)$. The detailed computational aspects are developed in Section \ref{sec:computation-revisited}.

\subsection{Summary and Outlook}

This section has introduced the generating functional framework that unifies computation, logic, and physics. The key elements are:

\begin{enumerate}
\item The sextary connective $\mathsf{G}_6$ provides the logical primitive whose denotation is the generating function
\item The generating function $G(z, \bar{z}; \vec{q}, \Lambda)$ unifies the three computational paradigms
\item The four-step pipeline appears consistently across paradigms
\item The Virasoro algebra provides the mathematical foundation for computational dynamics
\end{enumerate}

The renormalisation group machinery that reveals the deep structural connections is developed in Sections \ref{sec:regulator-view} through \ref{sec:generalized-rg-flows}. The logic transformer, an arity $(2,2)$ operator that implements the renormalisation procedure, is constructed in Section \ref{sec:complete-renormalization}.

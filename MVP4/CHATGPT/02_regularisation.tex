% Definition of Regularisation
% Extracted from ChatGPT conversation on renormalization logic

\section{The Regulator View: Understanding Compilation}
\label{sec:regulator-view}

The normalization step in the computational pipeline introduced in Section \ref{sec:S01_computation_invitation} corresponds to compilation, where regularisation makes the formal expansions well-defined. This section develops the regulator isomorphism that provides the mathematical foundation for understanding computation.

\subsection{The Regulator View of Computation}

\begin{definition}[Regulator View of Computation]
\label{def:regulator-view}
The computational process encoding → operator application → normalization (regularisation) → decoding corresponds to compilation, where:
\begin{align}
\text{Encoding} &: \text{Compile-time: input } (a,b) \mapsto z^a \bar{z}^b \\
\text{Application} &: \text{Runtime: repeated Virasoro operations under } \hat{H}_{\text{comp}} \\
\text{Normalization} &: \text{Regulator choice: boundaries, scale } \Lambda, \text{ grading parameters } \vec{q} \\
\text{Decoding} &: \text{Output: result extraction}
\end{align}
\end{definition}

\subsection{Natural Boundaries and Regulators}

\begin{definition}[Natural Computational Boundaries]
\label{def:natural-boundaries}
The framework has two natural boundaries:
\begin{align}
\text{Boundary at 0} &: \text{Empty computation, identity operation} \\
\text{Boundary at } \infty &: \text{Universal computation, all possible operations}
\end{align}
\end{definition}

The boundary at infinity serves as the overall regulator. Additional regulators emerge from the generating function structure.

\subsection{Regulator Hierarchy}

\begin{remark}[Regulator Hierarchy]
\label{rem:regulator-hierarchy}
The regulator structure follows a natural hierarchy that implements the main framework's moduli space structure:
\begin{align}
\text{Conceptual regulator} &: \text{Boundary at } \infty \\
\text{Operational regulator} &: \text{Overall scale parameter } \Lambda \text{ (1 overall coupling strength)} \\
\text{Grading parameters (the regulator trio)} &: \text{Grading parameters } \vec{q} = (q_1, q_2, q_3) \text{ (3 grading structure)} \\
\text{State regulator} &: \text{Virasoro levels } n, m
\end{align}
\end{remark}

\subsection{Termination Condition as Regularisation}

\begin{definition}[Termination Observable]
\label{def:termination-observable}
Let $P_{\mathrm{vac}}:=|0\rangle\langle 0|$. A run starting at $|\psi_0\rangle$ with $|\psi(t)\rangle=e^{-it\hat H}|\psi_0\rangle$ halts when
\[
\langle \psi(t)|P_{\mathrm{vac}}|\psi(t)\rangle\ge \tau
\]
for a fixed $\tau\in(0,1]$.
\end{definition}

The termination condition effectively imposes a canonical normalization on the computational process. By requiring the projection onto the vacuum subspace to exceed the threshold $\tau$, we ensure that the computational state has evolved sufficiently close to a well-defined reference state, providing a natural stopping criterion that is independent of the specific computational paradigm.


\subsection{Regularisation Principles}

The regularisation manifests differently across computational paradigms, but follows universal principles:
\begin{itemize}
\item \textbf{Boundary conditions} provide natural stopping criteria
\item \textbf{Normalization procedures} ensure well-defined results
\item \textbf{Scale parameters} control the computational process
\end{itemize}

Each paradigm implements these principles through different mathematical structures, unified by the generating function framework. The detailed paradigm-specific procedures are discussed in Section \ref{sec:complete-renormalization}.

\subsection{Connection to Traditional Computation}

The regulator view provides a natural bridge between traditional computational models and the generating function approach introduced in Section \ref{sec:S01_computation_invitation}. The key insight is that the normalization step corresponds to choosing appropriate regulators that make the computational process well-defined, unifying compile-time and runtime aspects through the generating function structure.
